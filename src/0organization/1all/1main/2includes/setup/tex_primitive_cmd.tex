%
%________________________________________________________________________
%========================================================================
% For Using "String-based Variables, collected under some Domain"
%  USAGE:
%         \setCollectVar{someCollection}{Var1}{12pt}
%         Value is: \getCollectVar{someCollection}{Var1}
%         \destroyCollectVar{someCollection}{Var1}
\makeatletter%
\newcommand{\setCollectVar}[3]{%
	\expandafter\def\csname #1@#2\endcsname{#3}%
}%
\newcommand{\getCollectVar}[2]{%
	\@ifundefined{#1@#2}{%
		% Handle the case where the variable is not defined
		\PackageError{DenKr-Variable-Collection}{#1@#2 is not defined}{}%
	}{%
		\csname #1@#2\endcsname%
	}%
}%
\newcommand{\destroyCollectVar}[2]{%
	\expandafter\let\csname #1@#2\endcsname\relax%
}%
\makeatother%
%------------------------------------------------------------------------
%========================================================================
%
%
%
%________________________________________________________________________
%========================================================================
% Compare a "\DenKr"-Variable with a Value (String)
% ARG:
%  1 - The Variable
%  2 - The Value to Compare with
%  3 - If equal (then do this)
%  4 - else (do this)
\newcommand{\DenKrCmp}[4]{%
	\expandafter\ifstrequal\expandafter{#1}{#2}{%
		#3%
	}{%
		#4%
	}%
}%
%  Ah, well, I know, "ifstrequal" isn't really a tex-primitive... But I always include 'etoolbox' very early on, as one of the first things anyway.
%________________________________________________________________________
%========================================================================
% An actually more tex-primitive Solution:
\newcommand{\DenKrITEequal}[4]{%
	\edef\argI{#1}%
	\edef\argII{#2}%
	\ifx\argI\argII%
		#3%
	\else%
		#4%
	\fi%
}%
%------------------------------------------------------------------------
%========================================================================
%
%
%
%________________________________________________________________________
%========================================================================
% Automatically escapes the underscore '_' for text inside it
%   USAGE: \escapeus{text_with_underscore}
\makeatletter%
\DeclareRobustCommand*{\escapeus}[1]{%
  \begingroup\@activeus\scantokens{#1\endinput}\endgroup%
}%
\begingroup\lccode`\~=`\_\relax%
   \lowercase{\endgroup\def\@activeus{\catcode`\_=\active \let~\_}}%
\makeatother%
%------------------------------------------------------------------------
%========================================================================
%
%
%
%________________________________________________________________________
%========================================================================
% A Macro for reading a File's Content to a Macro.
\newcommand{\ReadFileToMacro}[2]{%
	\providecommand{#1}{}%
	\IfFileExists{#2}{%
		\endlinechar=-1\relax%
		\newread\file%
		\openin\file=#2%
		\read\file to\DenKrNestedSrcDirTmpfile%
		\closein\file%
		\edef\DenKrNestedSrcDirTmpfile{\DenKrNestedSrcDirTmpfile}%
		\endlinechar=13\relax%
	}{}%
}%
% USAGE:
%    \newcommand{\CustomTempFile}{../tmp/DenKrNestedSrcDir.tex}%
%    \ReadFileToMacro{\DenKrNestedSrcDirTmpfile}{\CustomTempFile}%
%------------------------------------------------------------------------
%========================================================================
%
%
%
%________________________________________________________________________
%========================================================================
% Detecting the running TeX Engine
\makeatletter%
\newif\if@luatex\@luatexfalse%
\newif\if@pdftex\@pdftexfalse%
\newif\if@xetex\@xetexfalse%
\newif\if@ptex\@ptexfalse%
\newif\if@uptex\@uptexfalse%
\ExplSyntaxOn%
	\sys_if_engine_luatex:T{\@luatextrue}%
	\sys_if_engine_pdftex:T{\@pdftextrue}%
	\sys_if_engine_xetex:T{\@xetextrue}%
	\sys_if_engine_ptex:T{\@ptextrue}%
	\sys_if_engine_uptex:T{\@uptextrue}%
\ExplSyntaxOff%
\if@luatex%
	\typeout{»DenKr: TeX Engine in Use: LuaTeX}%
\else\if@pdftex%
	\typeout{»DenKr: TeX Engine in Use: pdfTeX}%
\else\if@xetex%
	\typeout{»DenKr: TeX Engine in Use: XeTeX}%
\else\if@ptex%
	\typeout{»DenKr: TeX Engine in Use: pTeX}%
\else\if@uptex%
	\typeout{»DenKr: TeX Engine in Use: upTeX}%
\fi\fi\fi\fi\fi%
% More similar stuff
\ifx\numexpr\undefined\else\typeout{»DenKr: eTex is activated}\fi%
\ifx\mubyte\undefined\else\typeout{»DenKr: encTeX is activated}\fi%
\newif\ifpdftex\newif\ifxetex\newif\ifluatex%
%\ifx\directlua\undefined\else\luatextrue\typeout{»DenKr: TeX Engine in Use: LuaTeX}\fi%
%\ifx\pdftexversion\undefined\else\pdftextrue\typeout{»DenKr: TeX Engine in Use: pdfTeX}\fi%
%\ifx\XeTeXrevision\undefined\else\xetextrue\typeout{»DenKr: TeX Engine in Use: XeTeX}\fi%
%\ifluatex\else\chardef\outputmode=0\fi%
%\ifpdftex\let\outputmode=\pdfoutput\fi%
%\ifxetex\chardef\outputmode=1\fi%
%\ifnum\outputmode>0 PDF is created\else DVI is created\fi%
%
%
% Set a 'String-Value' to the Compiler that is used:  LuaLaTeX, pdfLaTeX, XeLaTeX
\def\DenKrCompilerVALLua{LuaLaTeX}%
\def\DenKrCompilerVALPdf{pdfLaTeX}%
\def\DenKrCompilerVALXe{XeLaTeX}%
\newcommand{\defCompilerLua}{\xdef\DenKrCompiler{\DenKrCompilerVALLua}}%
\newcommand{\defCompilerPdf}{\xdef\DenKrCompiler{\DenKrCompilerVALPdf}}%
\newcommand{\defCompilerXe}{\xdef\DenKrCompiler{\DenKrCompilerVALXe}}%
%
% A Macro that executed Code depending on the TeX Engine in use.
%   I checks for the above created If
\newcommand{\DoIfLuaPdfXe}[3]{%
	%\ifthenelse{\boolean{@luatex} \OR \boolean{@xetex}}{}{}
	\if@luatex%
		#1%
	\else\if@pdftex%
		#2%
	\else\if@xetex%
		#3%
	\fi\fi\fi%
}%
\makeatother%
%
\DoIfLuaPdfXe{%
	\defCompilerLua%
}{%
	\defCompilerPdf%
}{%
	\defCompilerXe%
}%
%------------------------------------------------------------------------
%========================================================================