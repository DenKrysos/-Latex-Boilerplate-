% - For typesetting Listings slightly differently
%    (Don't load the packages, but insert a \fontfamily{}\selectfon in the \basicstyle argument
%    \usepackage[scaled=0.88]{beraserif}% \fontfamily{fvm}
%    \usepackage[scaled=0.85]{berasans}% \fontfamily{fvm}
%    \usepackage[scaled=0.84]{beramono}% \fontfamily{fvm}
% - Commands for switching to Bera Font
\makeatletter
\newcommand\BeraSerifsefamily{%
	\def\fve@Scale{0.88}% scales the font down
	\fontfamily{fve}\selectfont% selects the Bera font
}
\newcommand\BeraMonottfamily{%
	\def\fvm@Scale{0.84}%
	\fontfamily{fvm}\selectfont%
}
\newcommand\BeraSanssffamily{%
	\def\fvs@Scale{0.85}%
	\fontfamily{fvs}\selectfont%
}
\makeatother
% NOTE: That is currently not in use
%
%
% Define some special Signs
% - Lesser Equal with the the bottom being straight, i.e. not slanted
% - Greater Equal with the the bottom being straight, i.e. not slanted
%%%%%%%%%%%%%%%%%%%%%%%%%%%%%%%%%%%%%%%%%%%%%%%%%%%%%%%
% % % % libertinust1math.sty
\DeclareFontEncoding{LS2}{}{\noaccents@}%
\DeclareFontSubstitution{LS2}{libertinust1mathsym}{m}{n}%
\DeclareSymbolFont{libertinmath}{LS2}{libertinust1mathsym}{m}{n}%
% \DeclareSymbolFont{AMSb}{U}{msb}{m}{n}
\DeclareMathSymbol{\leqstraight}{3}{libertinmath}{"77}%
\DeclareMathSymbol{\geqstraight}{3}{libertinmath}{"78}%
% \DeclareMathSymbol{=}{\mathrel}{operators}{`=}
% - - - - - - - - - - - -
% % % Alternative (worse)
% \DeclareFontFamily{U}{matha}{\hyphenchar\font45}%
% \DeclareFontShape{U}{matha}{m}{n}{%
% 	<5> <6> <7> <8> <9> <10> gen * matha%
% 	<10.95> matha10 <12> <14.4> <17.28> <20.74> <24.88> matha12%
% }{}%
% \DeclareSymbolFont{matha}{U}{matha}{m}{n}%
% \DeclareFontSubstitution{U}{matha}{m}{n}%
% \DeclareMathSymbol{\leqstraight}{3}{matha}{"A8}%
% \DeclareMathSymbol{\geqstraight}{3}{matha}{"A9}%
%  %  %  %  %  %  %  %  %
\DeclareRobustCommand\qedsymbol{\ensuremath{\symup{∎}}}%
\let\qedsym\qedsymbol%
%%%%%%%%%%%%%%%%%%%%%%%%%%%%%%%%%%%%%%%%%%%%%%%%%%%%%%%