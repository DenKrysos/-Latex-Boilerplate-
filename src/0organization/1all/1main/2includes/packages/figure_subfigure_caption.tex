%________________________________________________________________________
%------------------------------------------------------------------------
%					Subfigure, Subfig, Subcaption, Caption
%/\/\/\/\/\/\/\/\/\/\/\/\/\/\/\/\/\/\/\/\/\/\/\/\/\/\/\/\/\/\/\/\/\/\/\/\
% In my personal opinion, package subcaption.sty (part of caption.sty) is preferable
%  While subfigure.sty is pretty much end-of-life, subfig.sty & subcaption.sty do both have their right to exist.
%  But in my personal opinion, subcaption.sty is superior in more regards than subfig.sty is
%
%           % \usepackage{subfigure}% Just as Info: subfigure is dead. Use subfig.sty (which would be loaded above by the IEEE-template) or subcaption.sty instead.
%           % \usepackage[caption=false,font=normalsize,labelfont=sf,textfont=sf]{subfig}
\usepackage{caption}%
\makeatletter%
\@ifundefined{DenKrCaptionStyPrep}{}{\DenKrCaptionStyPrep}%
\makeatother%
\usepackage{subcaption}%You can pass options for subcaptions/subfigures here or alternatively, use the following command
%
% - - - Values for the font-options
% - Size
%        tiny, scriptsize, footnotesize, small, normalsize, large, Large
% - Format
%        normalfont  --  Normal shape & series & family
%        up  --  Upright shape
%        it  --  Italic shape
%        sl  --  Slanted shape
%        sc  --  SMALL CAPS SHAPE
%        md  --  Medium series
%        bf  --  Bold series
%        rm  --  Roman family
%        sf  --  Sans Serif family
%        tt  --  Typewriter family
%/\/\/\/\/\/\/\/\/\/\/\/\/\/\/\/\/\/\/\/\/\/\/\/\/\/\/\/\/\/\/\/\/\/\/\/\
%					Subfigure, Subfig, Subcaption, Caption fertig
%------------------------------------------------------------------------
%________________________________________________________________________
%
%
%________________________________________________________________________
%------------------------------------------------------------------------
%				Settings
%/\/\/\/\/\/\/\/\/\/\/\/\/\/\/\/\/\/\/\/\/\/\/\/\/\/\/\/\/\/\/\/\/\/\/\/\
%  Asterisk (*) after \captionsetup suppresses the "Unused \captionsetup[⟨type⟩]" Warning
\captionsetup{%
	singlelinecheck=true,% If caption fits into 1-Line, always centered.
	format=plain,% plain, hang
	indention=1.25em,%
	justification=RaggedRight,% justified, centering, centerlast, centerfirst, raggedright, RaggedRight, raggedleft
	font=footnotesize,%
	labelfont=normalfont,%
	textfont=normalfont%
}%
% \captionsetup*[figure]{%
% 	font=footnotesize,%
% 	labelfont=normalfont,%
% 	textfont=normalfont%
% }%
%\captionsetup[subfigure]{margin=0pt,font+=smaller,labelformat=parens,labelsep=space,skip=6pt,list=false,hypcap=false}%Default options for subfigures
%\captionsetup{belowskip=0pt}% 12pt
%\captionsetup{aboveskip=4pt}% 4pt
%
% - - - -
%- Separating Space of Floats
%- -  Default Values of 'article' class with the '10pt' option 
% \setlength{\intextsep}{10pt plus 1.0pt minus 2.0pt}% stance between floats inserted inside the page text (using h) and the text proper
% \setlength{\textfloatsep}{10pt plus 1.0pt minus 2.0pt}% distance between floats on the top or the bottom and the text
% \setlength{\floatsep}{10pt plus 1.0pt minus 2.0pt}% distance between two floats
%
%\setlength{\intextsep}{6pt}%
%\setlength{\textfloatsep}{\intextsep}%