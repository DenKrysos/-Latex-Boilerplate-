%________________________________________________________________________
%------------------------------------------------------------------------
%           Typesetting / Schriftsatz / Satz (Druck)
%/\/\/\/\/\/\/\/\/\/\/\/\/\/\/\/\/\/\/\/\/\/\/\/\/\/\/\/\/\/\/\/\/\/\/\/\
%
% - - - - - - - -
% Text Alignment
% - - - - - - - -
\usepackage{ragged2e}%
\usepackage{tabto}%
%
\usepackage{wrapfig}% Allows to have a figure with text wrapped around
\usepackage{varwidth}% Cmds for textsize relative to surrounding text (as opposed to always absolut) -> (\smaller vs \small)
%
%----------------------------
%       Visuals
% - - - - - - - - - - - - - -
\usepackage{tcolorbox}%
\tcbuselibrary{skins}%
\tcbuselibrary{breakable}%
\tcbset{enhanced}%
%
%
%
%
%
%
%________________________________________________________________________
%------------------------------------------------------------------------
%           More matching "Typography"
%               (but common for Article & Book level classes, hence in this shared file)
% - - - - - - - - - - - - - -
\usepackage{textcomp}% LaTeX support for the Text Companion fonts
\usepackage{ulem}\normalem%
\usepackage{relsize}% Set the font size relative to the current font size
\usepackage{xspace}% Define commands that appear not to eat spaces
\usepackage{soul}% Hyphenation for letterspacing, underlining, and more
\usepackage{setspace}%
% - - - - - - - - - - - - - -
% Creating an Outline/Contour around Text
\newcommand{\DenKrOutlineWidth}{0.03em}%
\providecommand{\DenKrCompiler}{pdfLaTeX}%
\expandafter\ifstrequal\expandafter{\DenKrCompiler}{pdfLaTeX}{%
	\usepackage[pdftex,outline]{contour}%
	\contourlength{\DenKrOutlineWidth}% Thickness of the Outline. Default: 0.03em. Not expanded on Definition, but on Use.
	\contournumber{25}% How many copies are created. Default: 16
	%USAGE Example:
	% {%
	% 	\LARGE%
	% 	\contour{violet}{\textcolor{orange}{Text with Outline/Contour}}\nl%
	% 	{%
	% 		\contourlength{0.2em}%
	% 		\contour{violet}{\textcolor{orange}{Text with Outline/Contour}}\nl%
	% 	}%
	% }%
}{\expandafter\ifstrequal\expandafter{\DenKrCompiler}{LuaLaTeX}{%
	% Two different Packages and by that Methods are available:
	%  "contour": Prints the actual text as normal and surrounds it with the Contour.
	%  "pdfrender": Prints the outline as the actual text and fills this inside with a different Color.
	% -> "contour" grows bigger to the outside and makes the Outline sleeked (when growing too big)
	% -> "pdfrender" keeps the outline sharp and well defined, but may let the inside look wonky
	% => Methods to be ideally employed depends on the use-case
	%
	%Same Packge as for pdfLaTeX.
	\usepackage[outline]{contour}%
	\contourlength{\DenKrOutlineWidth}%
	\contournumber{25}%
	%
	% An alternative for LuaLaTeX. Works differently, but compiles more efficient.
	\usepackage{pdfrender}%
	%USAGE Example:
	% - The logic is kind-of inverted here.
	%   - What is set as the actual 'textcolor' is actually reflected as the Color of the Contour
	%   - The \textpdfrender is "filling" this inside.
	% \definecolor{CharFillColor}{rgb}{1,.8,.8}
	% \colorlet{CharFillColor}{orange}
	% \begin{document}
	% 	\sffamily
	% 	{\pdfrender{%
	% 		TextRenderingMode=FillStroke,
	% 		LineWidth=\DenKrOutlineWidth,
	% 		FillColor=CharFillColor,
	% 		StrokeColor=violet,
	% 		MiterLimit=1,
	% 	}Text with Outline/Contour}\nl%
	% 	{\pdfrender{%
	% 		TextRenderingMode=FillStroke,
	% 		LineWidth=0.06em,
	% 		FillColor=CharFillColor,
	% 		StrokeColor=violet,
	% 		MiterLimit=1,
	% 	}Text with Outline/Contour}%
	% \end{document}
}{%
}}%
% - - - - - - - - - - - - - -
% Additional Formatting-Cmds, letters, signs, special characters, units
% - - - - - - - - - - - - - -
%\usepackage{units}% Better don't use anymore. Instead siunitx.sty since it is newer and still maintained  %contains package 'nicefrac'.
\usepackage{nicefrac}% Supplies some nice Fraction formatting. \nicefrac{<Numerator>}{<Denominator>}
\usepackage{xfrac}% Another Fraction Formator. \sfrac[⟨instance⟩]{⟨num⟩}[⟨sep⟩]{⟨denom⟩}
% - - - SI Units - - - -
% When writing a fraction within a SI-Unit Statement, deactivate the number-parsing singular for this one occurence:
%   \num[parse-numbers=false]{\frac{7}{2}}
\usepackage[english,german,ngerman]{translator}%
\usepackage{siunitx}%
\sisetup{%
	input-ignore={,},%
	input-decimal-markers={.},%
	%output-decimal-marker={.},%
	%group-separator={,},%
	group-minimum-digits=5,% Determines the number of digits after which the grouping starts to be applied. Default: 5
	group-digits=all,%
	fraction-function=\nicefrac,% \dfrac
	range-phrase=-,%
	range-units=single%
}%
\DeclareSIUnit{\calorie}{cal}%
\DeclareSIUnit{\tablespoon}{\translate{tbsp}}%
\DeclareSIUnit{\teaspoon}{\translate{tsp}}%
\newtranslation[to=German]{tbsp}{EL}%
\newtranslation[to=German]{tsp}{TL}%
\newcommand{\groupSeparatorShortSpace}{\,}% \, is a "short-space", just a blank space, but shorter than a text-mode full space.
\newcommand{\groupSepGer}{.}%
\newcommand{\groupSepEn}{\groupSeparatorShortSpace}% Usually a 'short-space' or ',' (just plain comma)
\addto\extrasgerman{\sisetup{locale=DE,output-decimal-marker={,},group-separator={\groupSepGer},detect-all}}%
\addto\extrasenglish{\sisetup{locale=US,output-decimal-marker={.},group-separator={\groupSepEn},detect-all}}%
% For some reason, group-separator does not work in the addto configuration, hence adding it again below
\expandafter\ifstrequal\expandafter{\DenKrLanguageStringCmd}{ngerman}{%
	\sisetup{locale=DE,group-separator={\groupSepGer}}%
}{\expandafter\ifstrequal\expandafter{\DenKrLanguageStringCmd}{english}{%
	\sisetup{locale=US,group-separator={\groupSepEn}}%
}{}}%
%
%  Example:
%        \qty[locale=DE]{2}{\tablespoon}
%        %
%        \begin{otherlanguage}{english}
%        \qty{2}{\tablespoon}
%        \end{otherlanguage}
%        %
%        \begin{otherlanguage}{german}
%        \qty{2}{\tablespoon}
%        \end{otherlanguage}
%
% - - - - - - - - - -
% CAUTION: The Key-Definitions don't really work yet. Apparently, qty does prevent key expansion. I did not solve this yet.
% Some Formatting Helpers / Templates for the siunitx cmds
%  - Don't parse numbers
\newcommand{\SIrawNum}{parse-numbers=false}%
%       \qty[parse-numbers=false]{1.5}{\gram}  ->  \qty[\SIrawNum]{1.5}{\gram}
%  - State an "approximate" number. ->(siunitx supports so called comparators, such as <, =, > and also \approx and \sim)
%       \qty{\sim 3}{\gram}  instead of  \mytilde\qty{35}{\gram}
%  - Bold Printing  ->(Use these commands to replace hard-coding {\bfseries\boldmath\qty[detect-weight=true,detect-all=true]{500}{\ml} Glas}
\NewDocumentCommand{\qtybold}{omm}{%
	{\bfseries\boldmath%
	\IfNoValueTF{#1}{%
		\qty[detect-weight=true,detect-all=true]{#2}{#3}%
	}{%
		\qty[detect-weight=true,detect-all=true,#1]{#2}{#3}%
	}%
	}%
}%
\NewDocumentCommand{\qtyrangebold}{ommm}{%
	{\bfseries\boldmath%
	\IfNoValueTF{#1}{%
		\qtyrange[detect-weight=true,detect-all=true]{#2}{#3}{#4}%
	}{%
		\qtyrange[detect-weight=true,detect-all=true,#1]{#2}{#3}{#4}%
	}%
	}%
}%
%  - Group Digits (My default setting is to not start grouping right away after reaching the first 4. This directly groups with exceeding 3 digits)
\newcommand{\SIgroupDig}{group-minimum-digits=3}%
%       \qty[group-minimum-digits=3]{1500}{\liter}  ->  \qty[\SIgroupDig]{1500}{\liter}
% - - - - - - - - - -
% - - - - - - - - - -
% - - - - - - - - - -
% Two Macros for pasting time (time-of-day or time-span). Allow an optional first argument for local set up in the usual siunitx way.
\ExplSyntaxOn
% Hour-Minute-Second-Period (For pasting a Period of Time. Like \hmsP{10;20;30} -> "10:20:30 h" | \hmsP{;10;30}  ->  "20:30 min"
\NewDocumentCommand\hmsP{o >{\SplitArgument{2}{;}}m}{%
	\group_begin:%
		\IfNoValueF{#1}{%
			\keys_set:nn{siunitx}{#1}%
		}%
		\siunitx_hmsP_output:nnn #2%
	\group_end:%
}%
\cs_new_protected:Npn\siunitx_hmsP_output:nnn #1#2#3{%
	\qty[parse-numbers=false]{\mbox{
		\tl_if_blank:nF{#1}{\IfNoValueF{#1}{%
			#1%
			\tl_if_blank:nT{#2}{%
				\tl_if_blank:nF{#3}{\IfNoValueF{#3}{:}}%
			}\tl_if_blank:nF{#2}{\IfNoValueF{#2}{:}}%
		}}%
		\tl_if_blank:nF{#2}{\IfNoValueTF{#2}{%
			\tl_if_blank:nF{#1}{\IfNoValueF{#1}{\tl_if_blank:nF{#3}{\IfNoValueF{#3}{0:}}}}%
		}{%
			#2%
			\tl_if_blank:nF{#3}{\IfNoValueF{#3}{:}}%
		}}\tl_if_blank:nT{#2}{%
			\tl_if_blank:nF{#1}{\IfNoValueF{#1}{\tl_if_blank:nF{#3}{\IfNoValueF{#3}{0:}}}}%
		}%
		\tl_if_blank:nF{#3}{\IfNoValueF{#3}{%
			#3%
		}}%
	}}{%
		\tl_if_blank:nT{#1}{%
			\tl_if_blank:nT{#2}{%
				\tl_if_blank:nF{#3}{\IfNoValueF{#3}{\second}}%
			}\tl_if_blank:nF{#2}{\IfNoValueF{#2}{\minute}}%
		}\tl_if_blank:nF{#1}{\IfNoValueF{#1}{\hour}}%
	}%
}%
% Hour-Minute-Second-Clock (For pasting Time of Day / Clock Time. Like \hmsC{10;20;30} -> "10h 20 min 30s"
\NewDocumentCommand\hmsC{o >{\SplitArgument{2}{;}}m}{%
	\group_begin:%
		\IfNoValueF{#1}{%
			\keys_set:nn{siunitx}{#1}%
		}%
		\siunitx_hmsC_output:nnn #2%
	\group_end:%
}%
\cs_new_protected:Npn\siunitx_hmsC_output:nnn #1#2#3{%
	\IfNoValueF{#1}{%
		\tl_if_blank:nF{#1}{%
			\qty{#1}{\hour}%
			\IfNoValueF{#2}{~}%
		}%
	}%
	\IfNoValueF{#2}{%
		\tl_if_blank:nF{#2}{%
			\qty{#2}{\minute}%
			\IfNoValueF{#3}{~}%
		}%
	}%
	\IfNoValueF{#3}{%
		\tl_if_blank:nF{#3}{%
			\qty{#3}{\second}%
		}%
	}%
}%
\ExplSyntaxOff
% - - - - - - - - - - -
% Fraction Example with this:
%     $2\frac{7}{2}$, \nicefrac{7}{2}. \num[parse-numbers=false]{\frac{7}{2}}, \num[parse-numbers=false]{\nicefrac{1}{3}}.
%     \sfrac{1}{2}, $\sfrac{1}{2}$, $\mathbf{3\times\sfrac{1}{2}}$; \quad \fontfamily{ppl}\selectfont Palatino: \sfrac{1}{2}, \quad \fontfamily{ptm}\selectfont Times: \sfrac{1}{2}
% - - - - - - - -
%
%/\/\/\/\/\/\/\/\/\/\/\/\/\/\/\/\/\/\/\/\/\/\/\/\/\/\/\/\/\/\/\/\/\/\/\/\