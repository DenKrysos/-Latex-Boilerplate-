%________________________________________________________________________
%------------------------------------------------------------------------
%					Part
%/\/\/\/\/\/\/\/\/\/\/\/\/\/\/\/\/\/\/\/\/\/\/\/\/\/\/\/\/\/\/\/\/\/\/\/\
%
% own command to get a newline, without this disturbing underfull \hbox
% still beware of the "Glue" from the set \parskip height plus x minus y
% which in fact allows latex to stretch the space between paragraphs
% (by default up to 1pt)
\newcommand{\nl}{\par\noindent}% Abk. für NewLine
% and here a command to create really a new paragraph
% inserts an additional free line
\newcommand{\np}{\par\vspace{\baselineskip}}% Abk. für NewParagraph
\newcommand{\npi}{\par\vspace{\baselineskip}\noindent}% Abk. für NewParagraph_noIndent
% Zu benutzen also so:
% Neue Zeile ohne Einrückung: \nl
% Neue Zeile mit Einrückung: Freie Zeile im Editor
% Neuer Absatz mit freier Zeile dazwischen: \na
\newcommand{\greenuline}[1]{\colorlet{temp}{.}\color{green}\underline{\color{temp}#1}\color{temp}\xspace}%
%
%
%
%________________________________________________________________________
%------------------------------------------------------------------------
%					Part
%/\/\/\/\/\/\/\/\/\/\/\/\/\/\/\/\/\/\/\/\/\/\/\/\/\/\/\/\/\/\/\/\/\/\/\/\
% For creating a "Labeled Paragraph", which looks similar to a Description-List. This essentially just formats the given Text in a uniform way...
\newcommand{\parLabelBeforeskip}{0.34\baselineskip}%
\newcommand{\parLabelAfterlabelskip}{0.3em}%\enskip\quad\qquad%
\ExplSyntaxOn%
\DeclareDocumentCommand{\parLabel}{O{} m}{%
    \group_begin:%
    {%
    \keys_set:nn { parLabel } { #1 }
    \vspace{\parLabelBeforeskip}%
    \bfseries{\color{DenKrColor_LabeledParagraph}#2}\l_parLabel_delim_tl%
    }\hspace*{\parLabelAfterlabelskip}%
    \group_end:%
}%
% Define keys
\keys_define:nn { parLabel }{%
  delim .tl_set:N = \l_parLabel_delim_tl,%
  delim .initial:n = :,%
  % Add more keys as needed
}%
\ExplSyntaxOff%
%
%
%
%
%
%
%________________________________________________________________________
%------------------------------------------------------------------------
%			Referencing
%/\/\/\/\/\/\/\/\/\/\/\/\/\/\/\/\/\/\/\/\/\/\/\/\/\/\/\/\/\/\/\/\/\/\/\/\
% To do it *automatically*, using "Cleverref"
\newcommand{\entityref}[1]{\hyperref[#1]{\cref*{#1}}}%
\newcommand{\eleref}[1]{\entityref{#1}}%
\newcommand{\typeref}[1]{\entityref{#1}}%
% - - - - - - - - - - - - - - - -
% Including the *Name/Title/Caption* of the Element
\newcommand{\formatnamref}[1]{%
	%(#1)%
	$\langle$#1$\rangle$%
	%{\boldmath$\langle$}#1{\boldmath$\rangle$}%
	%$\lceil$#1$\rfloor$%
	%{\boldmath$$\lceil$}#1${\boldmath$\rfloor$}%
}%
\makeatletter%
\newcommand{\numnamref}[1]{%
    %\cref@gettype{#1}{\currlabtype}%
	\hyperref[#1]{\cref*{#1} \formatnamref{\nameref*{#1}}}%
}%
\makeatother%
\newcommand{\entitynumnamref}[1]{\numnamref{#1}}%
\newcommand{\elenumnamref}[1]{\numnamref{#1}}%
\newcommand{\typenumnamref}[1]{\numnamref{#1}}%
% - - - - - - - - - - - - - - - -
% Nameref with some fancy formatting
% - - Setup for it
\newcommand{\formatrefFormat}[1]{{\color{DenKrColor_Highlight_Emerald}#1}}%
\newcommand{\quotrefFormat}[1]{{\gerguiquotel{#1}}}%
\newcommand{\formatquotrefFormat}[1]{{\quotrefFormat{\formatrefFormat{#1}}}}%
% - - The actual referenc cmds
\newcommand{\formattednamref}[1]{{%
	\hyperref[#1]{\formatrefFormat{\nameref*{#1}}}%
}}%
\newcommand{\entityformattednamref}[1]{\formattednamref{#1}}%
\newcommand{\eleformattednamref}[1]{\formattednamref{#1}}%
\newcommand{\typeformattednamref}[1]{\formattednamref{#1}}%
%
\newcommand{\quotednamref}[1]{{%
	\hyperref[#1]{\quotrefFormat{\nameref*{#1}}}%
}}%
% - - - - - - - - - - - - - - - -
% The formatted above + quotation
\newcommand{\formatquotnamref}[1]{%
	\hyperref[#1]{\formatquotrefFormat{\nameref*{#1}}}%
}%
% - - - - - - - - - - - - - - - -
% For Referencing (additionally) deeper into a Chapter/Section
%    (Like "See Chapter 1.2 (NameOfIt) and therein the unnumbered SubSubSection ›Something‹)
%     -> e.g. \formatquotnamref{sec:someSec} (\furtherref{subsub:unnumberedSec})
%         --> becomes:  Section 1.2 (Section-Name) (→ ›NameOfUnnumberedSection‹)
\newcommand{\furtherrefFormat}[1]{{%
	$\rightarrow$ \formatquotnamref{#1}%
}}%
\newcommand{\furtherref}[1]{{%
	(\furtherrefFormat{#1})%
}}%
\newcommand{\deeperref}[1]{\furtherref{#1}}%
%
% - - - - - - - - - - - - - - - -
% To create a Reference with Custom-Text
% 1: The Hyperlink-Destination, i.e. just "sec:target" not "\pageref{sec:target}
% 2: The text that shall be displayed.
% -> The entire text is clickable and leads to the link's destination
\newcommand{\refCustom}[2]{%
    \hyperref[#1]{#2}%
}%
%
%
%
%
%
%
%
%________________________________________________________________________
%------------------------------------------------------------------------
%			Typesetting
%/\/\/\/\/\/\/\/\/\/\/\/\/\/\/\/\/\/\/\/\/\/\/\/\/\/\/\/\/\/\/\/\/\/\/\/\
% Usage: \DenKrSubfloat{<Label>}{<SubCaption>}{<FloatCode>}
%   Yes, I declared all three Arguments as mandatory, Deal with it.
\makeatletter\AddToHook{begindocument/before}{%
\@ifpackageloaded{subfig}{%
    \DeclareDocumentCommand{\DenKrSubfloat}{m m m}{%
        \subfloat[#2]{\label{#1}#3}%
    }%
}{\@ifpackageloaded{subcaption}{
    \DeclareDocumentCommand{\DenKrSubfloat}{m m m}{%
        \sbox{\tmpbox}{#3}%
        % \settowidth{\tmpwidth}{#3}%
        \settowidth{\tmpwidth}{\usebox{\tmpbox}}%
        \begin{minipage}{\tmpwidth}%
            \centering%
            \usebox{\tmpbox}%
            \protect\subcaption{#2}%
            \label{#1}%
        \end{minipage}%
    }%
}{\gerguiquote{Neither \textbf{subcaption} nor \textbf{subfig} loaded}}%
}%
}\makeatother%
%/\/\/\/\/\/\/\/\/\/\/\/\/\/\/\/\/\/\/\/\/\/\/\/\/\/\/\/\/\/\/\/\/\/\/\/\
%				Basic Typesetting END
%------------------------------------------------------------------------
%________________________________________________________________________
%
%
%
%
%
%________________________________________________________________________
%------------------------------------------------------------------------
%					Part
%/\/\/\/\/\/\/\/\/\/\/\/\/\/\/\/\/\/\/\/\/\/\/\/\/\/\/\/\/\/\/\/\/\/\/\/\
%
% Chapter with Subtitle
\newcommand\chaptersubt[2]{\chapter%
  [#1\hfil\hbox{}\protect\linebreak{\itshape#2}]%
  {#1\\[2ex]\Large\itshape#2}%
}%
\newcommand\sectionsubt[2]{\section%
  [#1\hfil\hbox{}\protect\linebreak{\itshape#2}]%
  {#1\\[2ex]\Large\itshape#2}%
}%
%
%
%
%
%
%
%
%
%
%
%
%
%
%
%
%
%
%
%
%
%________________________________________________________________________
%------------------------------------------------------------------------
%				Legacy / DEPRECATED
%/\/\/\/\/\/\/\/\/\/\/\/\/\/\/\/\/\/\/\/\/\/\/\/\/\/\/\/\/\/\/\/\/\/\/\/\
%
%________________________________________________________________________
% Referencing.
%   They should all be superseded by the other Referencing-Cmds above.
%   The ones, directly referencing more than one Element have still some value, though.
%   I know, I could (and should) extend the above macro to be able to take a SplitList. Someday...
%------------------------------------------------------------------------
%
% Eventuell Labelprüfung vor \cref@gettype einbauen:
%	(ifcsdef aus der etoolbox)
% Macros for uniform Formatting of References to Document-Elements (Figure, Equation, Algorithm, Table)
\newcommand{\abbref}[1]{\hyperref[#1]{Abbildung~\ref*{#1}}}%
\newcommand{\abbrefzwei}[2]{Abbildungen~\ref{#1} \& \ref{#2}}%
\newcommand{\figref}[1]{\hyperref[#1]{Figure~\ref*{#1}}}%
\newcommand{\figreftwo}[2]{Figures~\ref{#1} \& \ref{#2}}%
\newcommand{\absref}[1]{\hyperref[#1]{Abschnitt~\ref*{#1}}}%
\newcommand{\secref}[1]{\hyperref[#1]{Section~\ref*{#1}}}%
\newcommand{\tablref}[1]{\hyperref[#1]{Tabelle~\ref*{#1}}}%
\newcommand{\tabref}[1]{\hyperref[#1]{Table~\ref*{#1}}}%
\newcommand{\algorefGer}[1]{\hyperref[#1]{Algorithmus~\ref*{#1}}}%
\newcommand{\algoref}[1]{\hyperref[#1]{Algorithm~\ref*{#1}}}%
\newcommand{\gleref}[1]{\hyperref[#1]{Gleichung~\ref*{#1}}}%
\newcommand{\equref}[1]{\hyperref[#1]{Equation~\ref*{#1}}}%
% - - - - - - - - - - - - - - - -
%
% - - - - - - - - - - - - - - - -
% - Even more Legacy Referencing stuff
% - - (Better just ignore. Only still here, because I am not able to delete old code. '^.^  I might have a problem there...)
%--------------------------------
\newcommand{\abnamref}[1]{Abbildung~\ref{#1} (\nameref{#1})}%
\newcommand{\fignamref}[1]{Figure~\ref{#1} (\nameref{#1})}%
% \newcommand{\absref}[1]{\textbf{Abschnitt~\ref{#1}}}%
% Und das gleiche inklusive \nameref
\makeatletter%
\newcommand{\absamref}[1]{%
 	\cref@gettype{#1}{\currlabtype}%
	\IfSubStr{\currlabtype}{part}{%
		\textsf{\cref{#1} - \nameref{#1}}%
	}{%
	\IfSubStr{#1}{subsubsec}{%
%	'subsubsection' isn't functional with gettype.
%	This only goes deep until subsection
%	So be careful and append a preceding 'subsubsec'
%	on the label
		Unterabschnitt >>\nameref{#1}<< aus \ref{#1}%
	}{%
		\cref{#1} (\nameref{#1})%
	}%
	}%
}%
\makeatother%
\makeatletter%
\newcommand{\secnamref}[1]{%
 	\cref@gettype{#1}{\currlabtype}%
	\IfSubStr{\currlabtype}{part}{%
		\textsf{\cref{#1} - \nameref{#1}}%
	}{%
	\IfSubStr{#1}{subsubsec}{%
%	'subsubsection' isn't functional with gettype.
%	This only goes deep until subsection
%	So be careful and append a preceding 'subsubsec'
%	on the label
		Section >>\nameref{#1}<< of \ref{#1}%
	}{%
		\cref{#1} (\nameref{#1})%
	}%
	}%
}%
\makeatother%
% \newcommand{\absamref}[1]{\textbf{Abschnitt~\ref{#1}} (\nameref{#1})}%
% Für 2 Abschnitte zugleich
\makeatletter%
\newcommand{\absrefzwei}[2]{%
	\cref@gettype{#1}{\currlabtype}%
	\IfSubStr{\currlabtype}{chapter}{%
		Kapiteln~\ref{#1} \& \ref{#2}%
	}{%
		Abschnitten~\ref{#1} \& \ref{#2}%
	}%
}%
\makeatother%
\makeatletter%
\newcommand{\absamrefzwei}[2]{%
	\cref@gettype{#1}{\currlabtype}%
	\IfSubStr{\currlabtype}{chapter}{%
		Kapiteln~\ref{#1} (\nameref{#1}) \&%
			\ref{#2} (\nameref{#2})%
	}{%
		Abschnitten~\ref{#1} (\nameref{#1}) \&%
			\ref{#2} (\nameref{#2})%
	}%
}%
\makeatother%
% Absatz Name Reference Zwei Short -- means without the subsequent 'n' -
% i.e. Kapitel instead of Kapiteln
\makeatletter%
\newcommand{\absamrefzweis}[2]{%
	\cref@gettype{#1}{\currlabtype}%
	\IfSubStr{\currlabtype}{chapter}{%
		Kapitel~\ref{#1} (\nameref{#1}) \&%
			\ref{#2} (\nameref{#2})%
	}{%
		Abschnitte~\ref{#1} (\nameref{#1}) \&%
			\ref{#2} (\nameref{#2})%
	}%
}%
\makeatother%
% \newcommand{\absrefzwei}[2]{\textbf{Abschnitten~\ref{#1}} \& \ref{\ref{#2}}}%
% \newcommand{\absamrefzwei}[2]{\textbf{Abschnitten~\ref{#1}} (\nameref{#1}) \& \textbf{\ref{#2}} (\nameref{#2})}%
%
%
%
%
%
%
%________________________________________________________________________
% Actually, there is no longer a need for this (when Using in the Book-Boilerplate).
%  I've redefined the normal \chapter-Cmd to be way more powerful and by that, this also includes the here provided Functionality. Look into "0organization/1main/2includes/packages/typografie_redefine.tex"
%------------------------------------------------------------------------
% Makro to define a chapter without numbering, but with Entry in Table of Contents
% Chapter without numbering to TOC
% Arguments:
% #1 - Optional. Alternative Name for TOC
% #2 - Mandatory. Name of the Chap/Sec/Subsec itself
% #3 - Optional. Label
%
%					%				Old Version					%
% 					% 					% \newcommand{\chapnt}[2]{%
% 					% 					% 	\chapter*{#1}%
% 					% 					% 	\label{#2}%
% 					% 					% 	\addcontentsline{toc}{chapter}{\nameref{#2}}%
% 					% 					% 	\markboth{\nameref{#2}}{\nameref{#2}}%
% 					% 					% 	\addtocounter{chapter}{1}%
% 					% 					% }%
\DeclareDocumentCommand{\chapnt}{%
O{#2} m o%
}{%
	\chapter*{#2}%
	\IfValueTF{#3}{%
		\label{#3}%
	}{}%
% 	\addcontentsline{toc}{chapter}{\nameref{#3}}%
% 	\markboth{\nameref{#2}}{\nameref{#3}}%
	\addcontentsline{toc}{chapter}{#1}%
	\markboth{#1}{#1}%
	\addtocounter{chapter}{1}%
}%
% Usage: \chapnt{Chapter Name}{Label}
%
% And for Section
%	Sets the \rightmark. Remember to change the Brackets for \leftmark if needed
%	With empty Bracket it sets it blank
%					%				Old Version					%
% 					% 					% \newcommand{\secnt}[2]{%
% 					% 					% 	\section*{#1}%
% 					% 					% 	\label{#2}%
% 					% 					% 	\addcontentsline{toc}{section}{\nameref{#2}}%
% 					% 					% 	\markboth{}{\nameref{#2}}%
% 					% 					% }%
\DeclareDocumentCommand{\secnt}{%
O{#2} m o%
}{%
	\section*{#2}%
	\IfValueTF{#3}{%
		\label{#3}%
	}{}%
	\addcontentsline{toc}{section}{#1}%
% 	\markboth{\leftmark}{#1}%
	\markright{#1}%
}%
%
% And for SubSection
%					%				Old Version					%
% 					% 					% \newcommand{\subsecnt}[2]{%
% 					% 					% 	\subsection*{#1}%
% 					% 					% 	\label{#2}%
% 					% 					% 	%\addcontentsline{toc}{subsection}{\protect\numberline{}#1}%This would add a whitespace, where normally the section-numbering would be.
% 					% 					% 	\addcontentsline{toc}{subsection}{#1}%
% 					% 					% 	\markboth{}{\nameref{#2}}%
% 					% 					% }%
\DeclareDocumentCommand{\subsecnt}{%
O{#2} m o%
}{%
	\subsection*{#1}%
	\IfValueTF{#3}{%
		\label{#3}%
	}{}%
	%\addcontentsline{toc}{subsection}{\protect\numberline{}#1}%This would add a whitespace, where normally the section-numbering would be.
	\addcontentsline{toc}{subsection}{#1}%
}%