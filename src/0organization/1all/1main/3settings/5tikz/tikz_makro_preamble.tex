%%###########################################################%%
%%        vvvvvvvvvvvvvvvvvvvvvvvvvvvvvvvvvvvvvvvvvvv        %%
%%-----------------------------------------------------------%%
%  - Author: Dennis Krummacker
%%------------------------------------------------------------%%
%%        ^^^^^^^^^^^^^^^^^^^^^^^^^^^^^^^^^^^^^^^^^^^^        %%
%%============================================================%%
%
%
%
%========-------------------------------------------========
%========		General					========
%========-------------------------------------------========
% USAGE Example for following
%        \begin{tikzpicture}
%          \node(mynode)[text width=3cm, draw=red]{This is a node that varies in height and I need to know the hieght and store it in a length};
%          \getnodedimen{mynode}
%          \node[draw, minimum height=\nodeheight, minimum width=\nodewidth]at(4,0){};
%        \end{tikzpicture}
\NewDocumentCommand{\getnodedimen}{O{\nodewidth} O{\nodeheight} m}{%
  \begin{pgfinterruptboundingbox}%
  \begin{scope}[local bounding box=bb@temp]%
    \node[inner sep=0pt, fit=(#3)]{};%
  \end{scope}%
  \path($(bb@temp.north east)-(bb@temp.south west)$);%
  \end{pgfinterruptboundingbox}%
  \pgfgetlastxy{#1}{#2}%
}
%
\makeatletter%
\newcommand\tikzcurrentcoordinate{\the\tikz@lastxsaved,\the\tikz@lastysaved}%
\makeatother%
%
%
%
%========-------------------------------------------========
%========		Measurements and Text					========
%========-------------------------------------------========
% Explanation:
%    PGF/TikZ sets the "current font" inside a picture environment to \nullfont to ignore any text. Inside \node {..}; the font is then restored using \pgfinterruptpicture .. \endpgfinterruptpicture
%    In order to make something like \settowidth work, one needs to add \pgfinterruptpicture .. \endpgfinterruptpicture around the content. E.g.:
%       \settowidth{\somelength}{\pgfinterruptpicture <Text>\endpgfinterruptpicture}

%
%
%
%
%
%
%
%
%####################################################################
%====================================================================
%------------- Macro to calc the distance between two points in tikz -------------
%vvvvvvvvvvvvvvvvvvvvvvvvvvvvvvvvvvvvvvvvvvvvvvvvvvvvvvvvvvvvvvvvvvvv
\makeatletter%
\def\calcLength(#1,#2)#3{%
\pgfpointdiff{\pgfpointanchor{#1}{center}}%
             {\pgfpointanchor{#2}{center}}%
\pgf@xa=\pgf@x%
\pgf@ya=\pgf@y%
\FPeval\@temp@a{\pgfmath@tonumber{\pgf@xa}}%
\FPeval\@temp@b{\pgfmath@tonumber{\pgf@ya}}%
\FPeval\@temp@sum{(\@temp@a*\@temp@a+\@temp@b*\@temp@b)}%
\FProot{\FPMathLen}{\@temp@sum}{2}%
\FPround\FPMathLen\FPMathLen5\relax%
\global\expandafter\edef\csname #3\endcsname{\FPMathLen}%
}%
\makeatother%
%
% Usage Example
    % REMARK: The FPeval approach (i.e. the Macro "calcLength" is more precise
    %
        % \begin{tikzpicture}
        % \coordinate (A) at (1,2);
        % \coordinate (B) at (4,6);
        % \calcLength(A,B){mylen}
        % % \draw (A) circle (\mylen pt); % pt is important here
        % \end{tikzpicture}
        % With calclength the length of AB is : \mylen
        %
        % \begin{tikzpicture}
        % %  Two alternatives: Create an "Pseudo-Coordinate" (\p3) as subtraction and us its elements, or directly us the actual points in calculation (\n2 & \n4).
        % \coordinate(A)at(1,2);%
        % \coordinate(B)at(4,6);%
        % \path[draw=green]%
        %     let\p1=(A),\p2=(B),%
        %         \p3=($(B)-(A)$),% or: ($(\p2)-(\p1)$)
        %         \n1={veclen(\x3,\y3)},\n2={veclen(\x2-\x1,\y2-\y1)},%
        %         \n3={atan2(\y3,\x3)},\n4={atan2(\y2-\y1,\x2-\x1)}%
        %     in%
        %         (A)--(B)%
        %         node[draw,text=red,align=left]{Distance using veclen:\\\ \ \n1 / \n2\\Angle:\\\ \ \n3 / \n4}%
        % ;%
        % \end{tikzpicture}
%^^^^^^^^^^^^^^^^^^^^^^^^^^^^^^^^^^^^^^^^^^^^^^^^^^^^^^^^^^^^^^^^^^^^
%------------- tikz distance calc End -------------
%====================================================================
%####################################################################
%
%
%
%
%
%
%========-------------------------------------------========
%========		Remainder					========
%========-------------------------------------------========
%
% Extrusion Makros (needs shape=rectangleRoundedAnchorSurround)
% You have to specify a coordinate-Node as Base-Anchor and pass it
% Then pass the Directions
% Styles have to contain at least:
%	minimum width, minimum height, rounded corners attributes=RoundedCornersAmount
% Arguments:
% {AnchorNode}
% {up / down} %Not used for now
% {left / right} %Not used for now
% {StyleFront}
% {StyleBack}
% {FrontColor}
% {BackColor}
% {RoundedCornersAmount}
% Later on added:
% {ShiftAmountX}
% {SjoftAmountY}
%
\DeclareDocumentCommand{\ExtrudeOutRoundedTop}{ m m m m m m m m }{%
	\node(#1back)[#5,fill=#7,anchor=south]at(#1){};%
	\node(#1front)[#4,fill=#6,xshift=-2mm,yshift=2mm,anchor=south]at(#1){};%
	\path[draw=black,fill=#7](#1front.south west)--(#1front.south east)--%
		(#1back.south east)-- (#1back.south west)--(#1front.south west);%
	\path[shading=axis,%
		top color=#6,bottom color=#7,%
% 		middle color=#6,%
		shading angle=45,%
		transform canvas={shift={(-0.5\pgflinewidth,0)}},%
		]%
		($(#1front.east|-#1front.north)+(-#8,0)$)to[out=0,in=90]%
		($(#1front.east|-#1front.north)+(0,-#8)$)to%
		($(#1back.east|-#1back.north)+(0,-#8)$)to[out=90,in=0]%
		($(#1back.east|-#1back.north)+(-#8,0)$)to%
		($(#1front.east|-#1front.north)+(-#8,0)$);%
% 	\path[draw=black]%
% 		($(#1front.east|-#1front.north)+(-#8,0)$)to[out=0,in=90]%
% 		($(#1front.east|-#1front.north)+(0,-#8)$);%
% 	\path[draw=black]%
% 		($(#1back.east|-#1back.north)+(0,-16pt)$)to%
% 		($(#1back.east|-#1back.north)+(0,-#8)$)to[out=90,in=-45]%
%  		(#1back.north east);%
% 		($(#1back.east|-#1back.north)+(-#8,0)$)to%
% 		($(#1back.east|-#1back.north)+(-16pt,0)$);%
	\node(#1front2)[#4,fill=#6,anchor=south]at(#1front.south){};%
}%
%
\DeclareDocumentCommand{\ExtrudeOutRoundedBottom}{ m m m m m m m m }{%
	\node(#1back)[#5,fill=#7,anchor=north]at(#1){};%
	\node(#1front)[#4,fill=#6,xshift=-2mm,yshift=2mm,anchor=north]at(#1){};%
	\path[fill=abifacebackcolor](#1front.north east)--%
		(#1back.north east)--%
		(#1back.north east-|#1front.north east)--%
		(#1front.north east);%
		\draw[draw=abifacebackcolor,line width=0.8pt]%
			($(#1back.north east)+(0,-0.25\pgflinewidth)$)--%
			($(#1back.north east-|#1front.north east)+(0,-0.25\pgflinewidth)$);%
		\draw(#1front.north east)--(#1back.north east);%
		\draw[](#1back.north east)--%
			($(#1back.north east)+(0,-0.6pt)$);%
	\path[shading=axis,%
		left color=abifacebackcolor,right color=abifacebackcolor,%
		middle color=abifacecolor,%
		shading angle=135,]%
		($(#1front.east|-#1front.south)+(0pt,#8)$)to[out=-90,in=0]%
		($(#1front.east|-#1front.south)+(-#8,0pt)$)to%
		($(#1back.east|-#1back.south)+(-#8,0pt)$)to[out=0,in=-90]%
		($(#1back.east|-#1back.south)+(0pt,#8)$)to%
		($(#1front.east|-#1front.south)+(0pt,#8)$);%
	\path[draw=black]%
		($(#1back.east|-#1back.south)+(-16pt,0pt)$)to%
		($(#1back.east|-#1back.south)+(-#8,0pt)$)to[out=0,in=-90]%
		($(#1back.east|-#1back.south)+(0pt,#8)$)to%
		($(#1back.east|-#1back.south)+(0pt,16pt)$);%
	\path[shading=axis,%
		bottom color=#6,top color=#7,%
% 		middle color=#6,%
		shading angle=-135,%
		transform canvas={shift={(0,0.5\pgflinewidth)}},%
		]%
		($(#1front.west|-#1front.south)+(0,#8)$)to[out=-90,in=0]%
		($(#1front.west|-#1front.south)+(#8,0)$)to%
		($(#1back.west|-#1back.south)+(#8,0)$)to[out=180,in=-90]%
		($(#1back.west|-#1back.south)+(0,#8)$)to%
		($(#1front.west|-#1front.south)+(0,#8)$);%
	\node(#1front2)[#4,fill=#6,anchor=north]at(#1front.north){};%
}%