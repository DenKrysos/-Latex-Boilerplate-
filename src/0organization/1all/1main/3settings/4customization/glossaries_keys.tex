%________________________________________________________________________
%------------------------------------------------------------------------
%					Custom Key Definitions
%/\/\/\/\/\/\/\/\/\/\/\/\/\/\/\/\/\/\/\/\/\/\/\/\/\/\/\/\/\/\/\/\/\/\/\/\%
% - For the DualEntry Cmd, to link the opposing entry in Acronyms or Glossaries
\glsaddkey*{crossRef}%Key Name
% {\glsentrytext{\glslabel}deutsch}%Default Value
{}%Default Value
{\glsentrycrossRef}%Use analog \glsentrytext
{\GlsentrycrossRef}%Use analog \Glsentrytext
{\glscrossRef}%Use analog \glstext
{\GlscrossRef}%Use analog \Glstext
{\GLScrossRef}%Use analog \GLStext%
%------------------------------------------------------------------------
%------------------------------------------------------------------------
% - LANGUAGE-specific Definitions
% - - USAGE: Consider these as 'Alternatives' to the actual Glossary-Definition. I.e. it wouldn't make sense in many cases to use multiple at a time.
% - -  (I.e. define the 'actual entry', i.e. the standard-keys in one language and then only use the lang-secific keys for other language(s))
% - - -  (E.g. don't use english- and german-specific simultaneously, but define the entry as e.g. in english and then only populate the alternative keys for german)
% - - Other than that, you could also of course just define different entries, one for each language, and use the "see" key to reference them.
%- - - - - - - - - - - - - - - - - - - - - - - - -
% - Some keys to define explicitly in DEUTSCH/GERMAN
% The Abbreviation / Short Form. E.g. 'WiP'
\glsaddkey*{textDE}%Key Name
{}%Default Value
{\glsentrytextDE}%Use analog \glsentrytext
{\GlsentrytextDE}%Use analog \Glsentrytext
{\glstextDE}%Use analog \glstext
{\GlstextDE}%Use analog \Glstext
{\GLStextDE}%Use analog \GLStext%
% The Expansion / Long Form / Full Form. E.g. 'Work in Progress'
\glsaddkey*{longDE}%Key Name
{}%Default Value
{\glsentrylongDE}%Use analog \glsentrytext
{\GlsentrylongDE}%Use analog \Glsentrytext
{\glslongDE}%Use analog \glstext
{\GlslongDE}%Use analog \Glstext
{\GLSlongDE}%Use analog \GLStext%
% The Description / Longer Text in a Glossary-Entry (not Acronym)
\glsaddkey*{descDE}%Key Name
{}%Default Value
{\glsentrydescDE}%Use analog \glsentrytext
{\GlsentrydescDE}%Use analog \Glsentrytext
{\glsdescDE}%Use analog \glstext
{\GlsdescDE}%Use analog \Glstext
{\GLSdescDE}%Use analog \GLStext%
%------------------------------------------------------------------------
% - Some keys to define explicitly in DEUTSCH/GERMAN
% The Abbreviation / Short Form. E.g. 'WiP'
\glsaddkey*{textEN}%Key Name
{}%Default Value
{\glsentrytextEN}%Use analog \glsentrytext
{\GlsentrytextEN}%Use analog \Glsentrytext
{\glstextEN}%Use analog \glstext
{\GlstextEN}%Use analog \Glstext
{\GLStextEN}%Use analog \GLStext%
% The Expansion / Long Form / Full Form. E.g. 'Work in Progress'
\glsaddkey*{longEN}%Key Name
{}%Default Value
{\glsentrylongEN}%Use analog \glsentrytext
{\GlsentrylongEN}%Use analog \Glsentrytext
{\glslongEN}%Use analog \glstext
{\GlslongEN}%Use analog \Glstext
{\GLSlongEN}%Use analog \GLStext%
% The Description / Longer Text in a Glossary-Entry (not Acronym)
\glsaddkey*{descEN}%Key Name
{}%Default Value
{\glsentrydescEN}%Use analog \glsentrytext
{\GlsentrydescEN}%Use analog \Glsentrytext
{\glsdescEN}%Use analog \glstext
{\GlsdescEN}%Use analog \Glstext
{\GLSdescEN}%Use analog \GLStext%
%------------------------------------------------------------------------
% To provide an "Alternative"
% The Abbreviation / Short Form. E.g. 'WiP'
\glsaddkey*{textAlt}%Key Name
{}%Default Value
{\glsentrytextAlt}%Use analog \glsentrytext
{\GlsentrytextAlt}%Use analog \Glsentrytext
{\glstextAlt}%Use analog \glstext
{\GlstextAlt}%Use analog \Glstext
{\GLStextAlt}%Use analog \GLStext%
% The Expansion / Long Form / Full Form. E.g. 'Work in Progress'
\glsaddkey*{longAlt}%Key Name
{}%Default Value
{\glsentrylongAlt}%Use analog \glsentrytext
{\GlsentrylongAlt}%Use analog \Glsentrytext
{\glslongAlt}%Use analog \glstext
{\GlslongAlt}%Use analog \Glstext
{\GLSlongAlt}%Use analog \GLStext%
% The Description / Longer Text in a Glossary-Entry (not Acronym)
\glsaddkey*{descAlt}%Key Name
{}%Default Value
{\glsentrydescAlt}%Use analog \glsentrytext
{\GlsentrydescAlt}%Use analog \Glsentrytext
{\glsdescAlt}%Use analog \glstext
{\GlsdescAlt}%Use analog \Glstext
{\GLSdescAlt}%Use analog \GLStext%
%------------------------------------------------------------------------
%------------------------------------------------------------------------
\glsaddkey*{subtitle}%Key Name
{}%Default Value
{\glsentrysubtitle}%Use analog \glsentrytext
{\Glsentrysubtitle}%Use analog \Glsentrytext
{\glssubtitle}%Use analog \glstext
{\Glssubtitle}%Use analog \Glstext
{\GLSsubtitle}%Use analog \GLStext%
%
% - Just some additional Keys
\glsaddkey*{url}%Key Name
{}%Default Value
{\glsentryurl}%Use analog \glsentrytext
{\Glsentryurl}%Use analog \Glsentrytext
{\glsurl}%Use analog \glstext
{\Glsurl}%Use analog \Glstext
{\GLSurl}%Use analog \GLStext%
%
\glsaddkey*{longpltwo}%Key Name
{}%Default Value
{\glsentrylongpltwo}%Use analog \glsentrytext
{\Glsentrylongpltwo}%Use analog \Glsentrytext
{\glslongpltwo}%Use analog \glstext
{\Glslongpltwo}%Use analog \Glstext
{\GLSlongpltwo}%Use analog \GLStext%
%------------------------------------------------------------------------
%------------------------------------------------------------------------
% For "Mathematical" SYMBOLS, i.e. where the 'name' is defined in Math-Mode, so that they also can be used in Math-Mode
%  - You define the 'name' so that it works in normal text mode. E.g. name={$\sigma$}
%  - You put the same value - but without Math-Mode - in symraw. E.g. symraw={\sigma}
% -> In normal text, you continue just like ordinary, e.g. using \gls{sigmaKey}
% -> Inside a math environment, you use a symraw-cmd, e.g. $\glssymraw{sigmaKey}$
\glsaddkey*{symraw}%Key Name
{}%Default Value
{\glsentrysymraw}%Use analog \glsentrytext
{\Glsentrysymraw}%Use analog \Glsentrytext
{\glssymraw}%Use analog \glstext
{\Glssymraw}%Use analog \Glstext
{\GLSsymraw}%Use analog \GLStext%
%/\/\/\/\/\/\/\/\/\/\/\/\/\/\/\/\/\/\/\/\/\/\/\/\/\/\/\/\/\/\/\/\/\/\/\/\
%					Custom Key Definitions END
%------------------------------------------------------------------------
%________________________________________________________________________
%
%
%
%
%
%
%
%
%
%
%________________________________________________________________________
%------------------------------------------------------------------------
%					Custom "Composite" Key Definitions
%/\/\/\/\/\/\/\/\/\/\/\/\/\/\/\/\/\/\/\/\/\/\/\/\/\/\/\/\/\/\/\/\/\/\/\/\%
% These are constructed automatically from the Keys defined above
%
%  TODO The composite stuff is not yet finished. For ME, look into: glossaries_keys__TODO.tex
%
%/\/\/\/\/\/\/\/\/\/\/\/\/\/\/\/\/\/\/\/\/\/\/\/\/\/\/\/\/\/\/\/\/\/\/\/\
%					Composite Keys DONE
%------------------------------------------------------------------------
%________________________________________________________________________