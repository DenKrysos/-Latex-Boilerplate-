% Author: Dennis Krummacker%
\ifdefined\DenKr\documentclass[tikz,fontsize=11pt,class=scrbook]{standalone}\begin{document}\else%
\input{"./2ProjectSetup".tex}%
\input{"\DenKrAllMainRootDirPATH/2includes/packages/preamble_pre".tex}%
\documentclass[tikz,fontsize=11pt,class=scrbook]{standalone}% I.e. the content from \input{"\DenKrLayoutMainRootDir/2layout/tikz_standalone/preamble_1_class".tex}%
\input{"\DenKrInternalLayoutRootDir/tikz_standalone/1TikzStandalonePicIncludeThis".tex}%
\DenKrTikzStandalonePre\fi%
%
%
%
\makeatletter
\newbox\tikz@sand@box
\newcount\tikz@scope@depth
\tikz@scope@depth111\relax
\def\scopenode[#1]#2{%
    \begin{pgfinterruptboundingbox}%
        \advance\tikz@scope@depth111\relax%
        % process the user option
        \begin{scope}[name=tempscopenodename,at={(0,0)},anchor=center,#1]%
            % try to extract positioning information: name, at, anchor
            \global\let\tikz@fig@name\tikz@fig@name%
            \global\let\tikz@node@at\tikz@node@at%
            \global\let\tikz@anchor\tikz@anchor%
        \end{scope}%
        \let\tikz@scopenode@name\tikz@fig@name%
        \let\tikz@scopenode@at\tikz@node@at%
        \let\tikz@scopenode@anchor\tikz@anchor%
        % try to typeset this scope
        % we only need bounding box information
        % the box itself will be discard
        \setbox\tikz@sand@box=\hbox{%
            \begin{scope}[local bounding box=tikz@sand@box\the\tikz@scope@depth,#1]%
                #2%
            \end{scope}%
        }%
        % goodbye. haha
        \setbox\tikz@sand@box=\hbox{}%
        % now typeset again
        \begin{scope}[local bounding box=\tikz@scopenode@name]%
            % use the bounding box information to reposition the scope
            \pgftransformshift{\pgfpointanchor{tikz@sand@box\the\tikz@scope@depth}{\tikz@scopenode@anchor}%
                               \pgf@x-\pgf@x\pgf@y-\pgf@y}%
            \pgftransformshift{\tikz@scopenode@at}%
            \begin{scope}[#1]%
                #2
            \end{scope}%
        \end{scope}%
        \pgfkeys{/pgf/freeze local bounding box=\tikz@scopenode@name}%
        \global\let\tikz@scopenode@name@smuggle\tikz@scopenode@name%
    \end{pgfinterruptboundingbox}%
    % make up the bounding box
    \path(\tikz@scopenode@name@smuggle.south west)(\tikz@scopenode@name@smuggle.north east);%
    % draw something, not necessary
    \draw[#1](\tikz@scopenode@name@smuggle.south west)rectangle(\tikz@scopenode@name@smuggle.north east);%
}

\tikz{
    \scopenode[name=A,at={(5,10)},right]{
        \scopenode[name=Aa,at={(0,3)},right]{
            \scopenode[name=Aa1,at={(2,2)},opacity=.4]{
                \fill(0,0)rectangle(1,1)rectangle(2,2)rectangle(3,3);
            }
            \scopenode[name=Aa2,at={(Aa1.east)},right,opacity=.7]{
                \fill(0,0)rectangle(1,1)rectangle(2,2)rectangle(3,3);
            }
            \scopenode[name=Aa3,at={(Aa2.east)},right,opacity=1]{
                \fill(0,0)rectangle(1,1)rectangle(2,2)rectangle(3,3);
            }
        }
        \scopenode[name=Ab,at={(Aa.south west)},below,red]{
            \scopenode[name=Ab1,at={(2,2)},opacity=.4]{
                \fill(0,0)rectangle(1,1)rectangle(2,2)rectangle(3,3);
            }
            \scopenode[name=Ab2,at={(Ab1.east)},right,opacity=.7]{
                \fill(0,0)rectangle(1,1)rectangle(2,2)rectangle(3,3);
            }
            \scopenode[name=Ab3,at={(Ab2.east)},right,opacity=1]{
                \fill(0,0)rectangle(1,1)rectangle(2,2)rectangle(3,3);
            }
        }
        \scopenode[name=Ac,at={(Ab.south east)},below,blue]{
            \scopenode[name=Ac1,at={(2,2)},opacity=.4]{
                \fill(0,0)rectangle(1,1)rectangle(2,2)rectangle(3,3);
            }
            \scopenode[name=Ac2,at={(Ac1.east)},right,opacity=.7]{
                \fill(0,0)rectangle(1,1)rectangle(2,2)rectangle(3,3);
            }
            \scopenode[name=Ac3,at={(Ac2.east)},right,opacity=1]{
                \fill(0,0)rectangle(1,1)rectangle(2,2)rectangle(3,3);
            }
        }
    }
    \scopenode[name=B,at={(A.south west)},below,line width=10.4]{
        \scopenode[name=Ba,at={(0,3)},right]{
            \scopenode[name=Ba1,at={(2,2)},opacity=.4]{
                \fill(0,0)rectangle(1,1)rectangle(2,2)rectangle(3,3);
            }
            \scopenode[name=Ba2,at={(Ba1.east)},right,opacity=.7]{
                \fill(0,0)rectangle(1,1)rectangle(2,2)rectangle(3,3);
            }
            \scopenode[name=Ba3,at={(Ba2.east)},right,opacity=1]{
                \fill(0,0)rectangle(1,1)rectangle(2,2)rectangle(3,3);
            }
        }
        \scopenode[name=Bb,at={(Ba.south west)},below,red]{
            \scopenode[name=Bb1,at={(2,2)},opacity=.4]{
                \fill(0,0)rectangle(1,1)rectangle(2,2)rectangle(3,3);
            }
            \scopenode[name=Bb2,at={(Bb1.east)},right,opacity=.7]{
                \fill(0,0)rectangle(1,1)rectangle(2,2)rectangle(3,3);
            }
            \scopenode[name=Bb3,at={(Bb2.east)},right,opacity=1]{
                \fill(0,0)rectangle(1,1)rectangle(2,2)rectangle(3,3);
            }
        }
        \scopenode[name=Bc,at={(Bb.south east)},below,blue]{
            \scopenode[name=Bc1,at={(2,2)},opacity=.4]{
                \fill(0,0)rectangle(1,1)rectangle(2,2)rectangle(3,3);
            }
            \scopenode[name=Bc2,at={(Bc1.east)},right,opacity=.7]{
                \fill(0,0)rectangle(1,1)rectangle(2,2)rectangle(3,3);
            }
            \scopenode[name=Bc3,at={(Bc2.east)},right,opacity=1]{
                \fill(0,0)rectangle(1,1)rectangle(2,2)rectangle(3,3);
            }
        }
    }
    \scopenode[name=C,at={(B.south west)},below,line width=20.4]{
        \scopenode[name=Ca,at={(0,3)},right]{
            \scopenode[name=Ca1,at={(2,2)},opacity=.4]{
                \fill(0,0)rectangle(1,1)rectangle(2,2)rectangle(3,3);
            }
            \scopenode[name=Ca2,at={(Ca1.east)},right,opacity=.7]{
                \fill(0,0)rectangle(1,1)rectangle(2,2)rectangle(3,3);
            }
            \scopenode[name=Ca3,at={(Ca2.east)},right,opacity=1]{
                \fill(0,0)rectangle(1,1)rectangle(2,2)rectangle(3,3);
            }
        }
        \scopenode[name=Cb,at={(Ca.south west)},below,red]{
            \scopenode[name=Cb1,at={(2,2)},opacity=.4]{
                \fill(0,0)rectangle(1,1)rectangle(2,2)rectangle(3,3);
            }
            \scopenode[name=Cb2,at={(Cb1.east)},right,opacity=.7]{
                \fill(0,0)rectangle(1,1)rectangle(2,2)rectangle(3,3);
            }
            \scopenode[name=Cb3,at={(Cb2.east)},right,opacity=1]{
                \fill(0,0)rectangle(1,1)rectangle(2,2)rectangle(3,3);
            }
        }
        \scopenode[name=Cc,at={(Cb.south east)},below,blue]{
            \scopenode[name=Cc1,at={(2,2)},opacity=.4]{
                \fill(0,0)rectangle(1,1)rectangle(2,2)rectangle(3,3);
            }
            \scopenode[name=Cc2,at={(Cc1.east)},right,opacity=.7]{
                \fill(0,0)rectangle(1,1)rectangle(2,2)rectangle(3,3);
            }
            \scopenode[name=Cc3,at={(Cc2.east)},right,opacity=1]{
                \fill(0,0)rectangle(1,1)rectangle(2,2)rectangle(3,3);
            }
        }
    }
%   \foreach\A in{A,B,C}{
%       \draw[yellow,bend right,line width=10](-10,15)to(\A.west);
%       \foreach\a in{a,b,c}{
%           \draw[green,line width=5](-10,15)--(\A\a.north);
%           \foreach\n in{1,2,3}{
%               \draw(-10,15)--(\A\a\n.east);
%           }
%       }
%   }
}
%
%
\ifdefined\DenKr\end{document}\else\DenKrTikzStandalonePost\fi%