
% \tikzabb%[tex]%
% {tikztest}%
% {%
% Tikz-Picture Caption.%
% }%
% {1}[1,1,!ht]%[0.55,1,!ht]%
% {fig:tikz_test}%



\section{Some Example Section Name}

Example-Citation:
\cite{DenKr_denkrement1_indeco}
% \nl%
% A custom citation command that puts information about the reference to the footnote on first occurrence:
% \citeff{DenKr_MigArb,DenKr_denkrement1_indeco}%

\npi%
Some special characters:
»«.
\nl%
\gerguiquote{Done with Macro}\nl
\enquote{Ordinary Quotation}\ \ \enquote*{Single Quotation}
\nl%
Example Acronym, Glossary-Entry \& Symbol:\nl
\gls{fsa}, \glsunset{api}\gls{api}\nl
\gls{molmass}, \gls{sigma}\nl
\gls{eventfd}

\np
\newcommand{\I}{\mathrm{i}}
$y = \int_0^x\cos(x)\,\mathrm{d}{x} = \frac{e^{\I x} - e^{-\I x}}{2\I} | 0123456789 \geq i \leqslant l$
\nl
\begin{equation}
y = \int_0^x\cos(x)\,\mathrm{d}{x} = \frac{e^{\I x} - e^{-\I x}}{2\I} | 0123456789 \geq i \leqslant l
\end{equation}

\npi%
Document-within References to check whether hyperref \& nameref work properly:\nl%
\ref{sec:intro} (\nameref{sec:intro}). \ref{sec:RelWork}  (\nameref{sec:RelWork}). \ref{sec:Concl} (\nameref{sec:Concl})\nl%
Consider using my \textbf{Referencing-Macros}:\nl
\eleref{fig:Pic}\nl
\ \ or \elenumnamref{tab:OI_app}\nl
\ \ or \eleref{fig:Pic,fig:tikz_test,fig:Pic}


\npi%
{%
	\LARGE%
	\contourlength{\DenKrOutlineWidth}% The Default-Value here
	\contour{violet}{\textcolor{orange}{Text with Outline/Contour}}\nl%
	{%
		\contourlength{0.2em}%
		\contour{violet}{\textcolor{orange}{Text with Outline/Contour}}\nl%
		\contour{violet}{\textcolor{orange}{Text with Outline/Contour}}\nl%
	}%
	\contour{violet}{\textcolor{orange}{Text with Outline/Contour}}\nl%
}%

\par\vspace{1.0\baselineskip}
\noindent Expanding outwards, featuring \textbf{linebreak}\nl%
{%
	\Large%
	\DenKrContour[0.06em]{violet}{orange}{A Macro of mine, to retain Text breakable with the outwards growing approach, as you can see here happening.}\par%
}%
\par



\nl%
%
\begin{figure}[!htpb]
	\centering
	\includestandalone[mode=\includestandalonedefaultmode]{\tikzFilesPath/tikztest}% width=\columnwidth
	\caption{Tikz-Picture Caption. Some Example Standalone-Tikz-Pic. Examples for Standalone-TikZ Picture Files can be found in \enquote{0organization/1main/8templates/tikz/7Tikz}.}
	\label{fig:tikz_test}
\end{figure}
%
%
% \tikzabb%[tex]%
% {tikztest}%
% [\DenKrLayoutMainRootDir/8templates/tikz/7Tikz]% Alternative Path to the std "./7Tikz"
% {%
% Tikz-Picture Caption. Some Example Standalone-Tikz-Pic
% }%
% {1}[1,1,!ht]%[0.55,1,!ht]%
% {fig:tikz_test2}%
%
%
\begin{figure}%[!ht]
\centering
	\includegraphics[width=0.3\linewidth]{{"\DenKrGraphicsRootDir/example_AGV"}.pdf}%
	\caption{Some another Pic}%
	\label{fig:Pic}
\end{figure}





\npi
Listing in separate File:
\DenKrLstInput[
	language=DenKr-C,
	morekeywords={[4]{%Typedefed, Names of Struct, Enum, ...
		some_struct
	}},
	morekeywords={[5]{%Collection-Members, Struct-Instances
		tim,
		entry,
		member,
		iterator
	}},
	morekeywords={[6]{%Variables
		current
	}},
]{\DenKrListingsRootDir/example_lst_C.c}
[Example-Listing (for Programming-Language \textit{C})]
[lst:exmplLabel]
%
%
\lstinputlisting[
	frame=single,
	label=lst:cfgCalc,
	caption={Example-Code-Listing (for Pseudo-Code)},
	captionpos=b,
	language=DenKr-JavaScript
]
{\DenKrListingsRootDir/example_lst.pseudo}

\npi
Listing in Environment, directly in Code:
\begin{DenKrLst}[language=DenKr-C]
if(1){
	strlen("Str");
}
\end{DenKrLst}

\npi
An Inline-Listing:
\lstinline[language=DenKr-C,breaklines=true,morekeywords={[4]{msghandle}}]$struct msghandle$



%
\input{"\DenKrTablesRootDir/example_table".tex}%
%