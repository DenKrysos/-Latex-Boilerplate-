%
%================================================================
%----------------------------------------------------------------
%      Setup, Typographisch
%             (Für Präambel. Titel, Verzeichnisse...)
%================================================================
%
\providecommand{\DenKrMicrotypeLoaded}{0}%
\expandafter\ifstrequal\expandafter{\DenKrMicrotypeLoaded}{1}{%
\microtypesetup{activate=false}%
}{}%
%%%%%%%%%%%%%%%%%%%%%%%%%%%%%%%%%%%%%%%%%%%%%%%%%%%%%%%%%%%%%%%%%
% Tracking, Spacing & Kerning werden seit Lua(La)Tex von 'fontspec' verwaltet
% 				% \microtypesetup{tracking=false}%
% 				% \microtypesetup{kerning=false}%
% 				% \microtypesetup{spacing=false}%
%
%
%
%================================================================
%----------------------------------------------------------------
%      Title Page (no Mock-Title)
%================================================================
\part{Front-Matter}%
%
\section*{Title-Page}%
\begin{frame}[plain]%
% \frametitle{Title}%
\titlepage%
\end{frame}%
% \pagestyle{plain}
%
%
%================================================================
%----------------------------------------------------------------
%      Front-Matter
%================================================================
% - *Optional* Abstract/Introduction/Prologue
\newcommand{\DenKrAbstractFile}{"\DenKrSegmentationSubDir/0100_abstract_slides".tex}%
\IfFileExists{\DenKrAbstractFile}{\input{\DenKrAbstractFile}}{}%
%
\section*{Table-Of-Contents}% The order of this section and part seems twisted, but is indeed deliberate. This way, you have a Bookmark in the pdf-Navigation, but the ToC does count towards the Front-Matter and does not show up in the on-slide navigation (depending on theme) of the Main-Matter.
\part{Main-Matter}%
\begin{frame}[plain,allowframebreaks]%
\frametitle{Outline}% ToC
\tableofcontents%
% \tableofcontents[pausesections]%
\end{frame}%
%
%
%================================================================
%----------------------------------------------------------------
%      Setup, Typographisch
%             (Für eigentliches Dokument)
%================================================================
%
%\sloppy % Würde die Dehnung der Zwischenräume etwas lockerer machen
%\fussy % Schaltet \sloppy wieder aus. Ist der Latex-Standart
\expandafter\ifstrequal\expandafter{\DenKrMicrotypeLoaded}{1}{%
\microtypesetup{activate=true}%
}{}%
%%%%%%%%%%%%%%%%%%%%%%%%%%%%%%%%%%%%%%%%%%%%%%%%%%%%%%%%%%%%%%%%%
% Tracking, Spacing & Kerning werden seit Lua(La)Tex von 'fontspec' verwaltet
% 					% \microtypesetup{tracking=true}%
% 					% \microtypesetup{kerning=true}%
% 					% \microtypesetup{spacing=true}%