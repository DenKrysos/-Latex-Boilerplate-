%________________________________________________________________________
%------------------------------------------------------------------------
%               File Split Notice
%/\/\/\/\/\/\/\/\/\/\/\/\/\/\/\/\/\/\/\/\/\/\/\/\/\/\/\/\/\/\/\/\/\/\/\/\
%   More inside "./0organization/1main/3settings/" in the File "settings_basic.tex"
%       (Linked/Shared for every Project Commonly from the "-Latex-Essentials-Setup_Setting_STY_AddOn" Project)
%/\/\/\/\/\/\/\/\/\/\/\/\/\/\/\/\/\/\/\/\/\/\/\/\/\/\/\/\/\/\/\/\/\/\/\/\
%------------------------------------------------------------------------
%________________________________________________________________________
%
%
%
%
%
%________________________________________________________________________
%------------------------------------------------------------------------
%		Spacing für
%					- Equation Environment
%					- floats
%					- Some new Lengths & Variables
%/\/\/\/\/\/\/\/\/\/\/\/\/\/\/\/\/\/\/\/\/\/\/\/\/\/\/\/\/\/\/\/\/\/\/\/\
%	Is found in
%		\input{./0organization/1main/settings_inDocument.tex}
% ( '\setlength' must come after \begin{document} )
%/\/\/\/\/\/\/\/\/\/\/\/\/\/\/\/\/\/\/\/\/\/\/\/\/\/\/\/\/\/\/\/\/\/\/\/\
%			Spacing End
%------------------------------------------------------------------------
%________________________________________________________________________
%
%
%
%
%
%________________________________________________________________________
%------------------------------------------------------------------------
%							Setups für Microtype
%/\/\/\/\/\/\/\/\/\/\/\/\/\/\/\/\/\/\/\/\/\/\/\/\/\/\/\/\/\/\/\/\/\/\/\/\
\providecommand{\DenKrMicrotypeLoaded}{0}%
\expandafter\ifstrequal\expandafter{\DenKrMicrotypeLoaded}{1}{%
\SetProtrusion{encoding={*},family={bch},series={*},size={6,7}}
              {1={ ,750},2={ ,500},3={ ,500},4={ ,500},5={ ,500},
               6={ ,500},7={ ,600},8={ ,500},9={ ,500},0={ ,500}}
\SetExtraKerning[unit=space]
    {encoding={*}, family={bch}, series={*}, size={footnotesize,small,normalsize}}
    {\textendash={400,400}, % en-dash, add more space around it
     "28={ ,150}, % left bracket, add space from right
     "29={150, }, % right bracket, add space from left
     \textquotedblleft={ ,150}, % left quotation mark, space from right
     \textquotedblright={150, }} % right quotation mark, space from left
\SetTracking{encoding={*}, shape=sc}{40}
}{}%
%/\/\/\/\/\/\/\/\/\/\/\/\/\/\/\/\/\/\/\/\/\/\/\/\/\/\/\/\/\/\/\/\/\/\/\/\
%							Microtype fertig
%------------------------------------------------------------------------
%________________________________________________________________________
%
%
%
%
%
%________________________________________________________________________
%------------------------------------------------------------------------
%		Box Typesetting, Line Breaking, Warnings, Severity
%			- Lines, linewidth, width, length, height, overhang
%			(overfull / underfull \hbox / \vbox. too wide, badness)
%/\/\/\/\/\/\/\/\/\/\/\/\/\/\/\/\/\/\/\/\/\/\/\/\/\/\/\/\/\/\/\/\/\/\/\/\
% You know, like when the elements to place into a line are so awkward (maybe because it has unbreakable elements near the end-of-line, hyphenation is not known or possible) that the setting of text cannot be done well and a bit of it protrudes beyond the actual line/page border.
%------------------------------------------------------------------------
% (I'd say, a bit of Protrusion can be tolerated. If you aren't screwing your document up real bad, insanely crucial cases can't arise anyway. Small ones, about up to 6pt (or rather ~0.6em) or so, could arise however. This won't be spotted by the eye if not actively looking for it. So I say, just surpress the Warnings in such insignificant cases.)
%------------------------------------------------------------------------
% <A bit of LaTeX explanation>:
%	\tolerance
%		Sets the maximum "badness" that tex is allowed to use while setting the paragraph. Changing this affects the Typesetting.
%		The text placement inserts breakpoints allowing white space to stretch and penalties to be taken, so long as the badness keeps below this threshold. If it can not do that then you get overfull boxes. So different values produce different typeset results.
%	\emergencystretch (added at TeX3)
%		is used if TeX can not set the paragraph below the \tolerance badness, but rather than make overfull boxes it tries an extra pass "pretending" that every line has an additional \emergencystretch of stretchable glue, this allows the overall badness to be kept below 1000 and stops TeX "giving up" and putting all stretch into one line. So \emergencystretch does not change the setting of "good" paragraphs, it only changes the setting of paragraphs that would have produced over-full boxes. Note that you get warnings about the real badness calculation from TeX even though it retries with \emergencystretch. The extra stretch is used to control the typesetting but it is not considered as good for the purposes of logging.
%	\hfuzz
%		Warning regarding "overfull"
%		Does not affect the typesetting in any way but just stops TeX complaining if the box is only over-full by the set amount.
%		- Alright should be about ~5pt, 0.5em - 0.7em, ~0.6em
%	\hbadness
%		Warning regarding "underfull"
%		Similar to \hfuzz, it does not affect the typesetting but stops TeX warning about underfull boxes if the badness is below the given amount.
%		(not used in \sloppy)
%------------------------------------------------------------------------
%\/\/\/\/\/\/\/\/\/\/\/\/\/\/\/\/\/\/\/\/\/\/\/\/\/\/\/\/\/\/\/\/\/\/\/\/
%		Box Typesetting Done
%------------------------------------------------------------------------
%________________________________________________________________________
% \hfuzz=0.6em%
% % \hbadness=500%
%
%
%
%
%
%________________________________________________________________________
%------------------------------------------------------------------------
%							Setups für enumitem
%					Lists - (itemize, enumerate, description)
%/\/\/\/\/\/\/\/\/\/\/\/\/\/\/\/\/\/\/\/\/\/\/\/\/\/\/\/\/\/\/\/\/\/\/\/\
%  More in the "_basic.tex" File inside "./0organization/1main/3settings/"
%    -> The "_perDoc.tex" comes after the "_basic.tex" and hence its Settings overwrite.
%
%---------------------------
% %  List-Style Settings
% - - - - - - - -
% \renewcommand{\DenKrDescriptionlabelFont}{\normalfont\bfseries}%
% \renewcommand{\DenKrDescriptionlabelFont}{\normalfont\color{DenKrColor_DescriptionLabel}\bfseries}%
\renewcommand{\DenKrDescriptionlabelFont}{\normalfont\itshape}%
% \renewcommand{\DenKrDescriptionlabelFont}{\normalfont\color{DenKrColor_DescriptionLabel}\itshape}%
%
\renewcommand{\DenKrDescriptionlabelFormat}[1]{#1:}%
%
% - - Templates
%       For Templates, have a look into the corresponding File inside the "8templates" Directory
%/\/\/\/\/\/\/\/\/\/\/\/\/\/\/\/\/\/\/\/\/\/\/\/\/\/\/\/\/\/\/\/\/\/\/\/\
%							enumitem fertig
%------------------------------------------------------------------------
%________________________________________________________________________
%
%
%
%
%________________________________________________________________________
%------------------------------------------------------------------------
%		Spacing für
%					- Equation Environment
%					- floats
%					- Some new Lengths & Variables
%/\/\/\/\/\/\/\/\/\/\/\/\/\/\/\/\/\/\/\/\/\/\/\/\/\/\/\/\/\/\/\/\/\/\/\/\
% - - - - - - - - - - - - - - - - - - - - - - - - - - - - - - - - - - - -
%- - - - - - - - - - - - - - - - - - - - - - - - - - - - - - - - - - - - -
% - - - - - - - - - - - - - - - - - - - - - - - - - - - - - - - - - - - -
%		Equation
% - - - - - - - - - - - - - - - - - - - - - - - - - - - - - - - - - - - -
%------------------------------------------------------------------------
% 	\begin{equation}\begin{split}
% 	MRes_{K}^{3rd} = \delta_{3rd} + \delta_{K}
% 	\end{split}\end{equation}
%------------------------------------------------------------------------
% Default for 'above' & 'below' Space:
%		11.0pt plus 3.0pt minus 6.0pt
% You can check this using (prints it into the Document):
%		\the\abovedisplayskip
%		\the\belowdisplayskip
%------------------------------------------------------------------------
% The ones with 'short' come in play if the last line immediately before an equation is, well, short.
%/\/\/\/\/\/\/\/\/\/\/\/\/\/\/\/\/\/\/\/\/\/\/\/\/\/\/\/\/\/\/\/\/\/\/\/\
\setlength{\abovedisplayskip}{1.1ex}
\setlength{\belowdisplayskip}{1.1ex}
\setlength{\abovedisplayshortskip}{0.3ex}
\setlength{\belowdisplayshortskip}{\belowdisplayskip}
% - - - - - - - - - - - - - - - - - - - - - - - - - - - - - - - - - - - -
%- - - - - - - - - - - - - - - - - - - - - - - - - - - - - - - - - - - - -
% - - - - - - - - - - - - - - - - - - - - - - - - - - - - - - - - - - - -
%		Floats
% - - - - - - - - - - - - - - - - - - - - - - - - - - - - - - - - - - - -
% \setlength\extrarowheight{1ex}%
% The Spacing between floats and the surrounding text at top and bottom of the float. The Defaults are somewhat complicated and depend on document type and font-size%
% \setlength{\intextsep}{\intextsep}%
% Some more lengths like this
%     \textfloatsep — distance between floats on the top or the bottom and the text;
%     \floatsep — distance between two floats;
%     \intextsep — distance between floats inserted inside the page text (using h) and the text proper.
% - - - - - - - - - - - - - - - - - - - - - - - - - - - - - - - - - - - -
%				More on Spacing around Floats
%/\/\/\/\/\/\/\/\/\/\/\/\/\/\/\/\/\/\/\/\/\/\/\/\/\/\/\/\/\/\/\/\/\/\/\/\
% You can modify the following lengths, which affect all floats.;
%    \floatsep: space left between floats (12.0pt plus 2.0pt minus 2.0pt).
%    \textfloatsep: space between last top float or first bottom float and the text (20.0pt plus 2.0pt minus 4.0pt).
%    \intextsep : space left on top and bottom of an in-text float (12.0pt plus 2.0pt minus 2.0pt).
%    \dbltextfloatsep is \textfloatsep for 2 column output (20.0pt plus 2.0pt minus 4.0pt).
%    \dblfloatsep is \floatsep for 2 column output (12.0pt plus 2.0pt minus 2.0pt).
%    \abovecaptionskip: space above caption (10.0pt).
%    \belowcaptionskip: space below caption (0.0pt).
%\/\/\/\/\/\/\/\/\/\/\/\/\/\/\/\/\/\/\/\/\/\/\/\/\/\/\/\/\/\/\/\/\/\/\/\/
%			Spacing End
%------------------------------------------------------------------------
%________________________________________________________________________
%