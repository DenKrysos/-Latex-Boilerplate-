%
%================================================================
%----------------------------------------------------------------
%      Setup, Typographisch
%             (Für Präambel. Titel, Verzeichnisse...)
%================================================================
%
\pagestyle{empty}%
% Hm, settings vlt. doch besser in Präambel
% \input{./organization/settings.tex}%
\providecommand{\DenKrMicrotypeLoaded}{0}%
\expandafter\ifstrequal\expandafter{\DenKrMicrotypeLoaded}{1}{%
\microtypesetup{activate=false}%
}{}%
%%%%%%%%%%%%%%%%%%%%%%%%%%%%%%%%%%%%%%%%%%%%%%%%%%%%%%%%%%%%%%%%%
% Tracking, Spacing & Kerning werden seit Lua(La)Tex von 'fontspec' verwaltet
% 				% \microtypesetup{tracking=false}%
% 				% \microtypesetup{kerning=false}%
% 				% \microtypesetup{spacing=false}%
%
%================================================================
%----------------------------------------------------------------
%      Title Page \& Schmutztitel
%================================================================
\input{"\DenKrLayoutRootDir/11_TitlePage".tex}%
%
%
%================================================================
%----------------------------------------------------------------
%      Front-Matter
%================================================================
% By default, no explicit Front-Matter in Article.
% \frontmatter%
\input{"\DenKrLayoutRootDir/12_FrontMatter".tex}%
%
%
%================================================================
%----------------------------------------------------------------
%      Setup, Typographisch
%             (Für eigentliches Dokument)
%================================================================
%
% \selectlanguage{ngerman}%
%
%\sloppy % Würde die Dehnung der Zwischenräume etwas lockerer machen
%\fussy % Schaltet \sloppy wieder aus. Ist der Latex-Standart
\expandafter\ifstrequal\expandafter{\DenKrMicrotypeLoaded}{1}{%
\microtypesetup{activate=true}%
}{}%
%%%%%%%%%%%%%%%%%%%%%%%%%%%%%%%%%%%%%%%%%%%%%%%%%%%%%%%%%%%%%%%%%
% Tracking, Spacing & Kerning werden seit Lua(La)Tex von 'fontspec' verwaltet
% 					% \microtypesetup{tracking=true}%
% 					% \microtypesetup{kerning=true}%
% 					% \microtypesetup{spacing=true}%
%
% Beachte: Typographisch erhalten Kapitelseiten keinen Kolumnentitel: plain
\pagestyle{scrheadings}%