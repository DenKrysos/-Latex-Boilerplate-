%========================================================================
%       Setting up some Macros for later use
% - - - - - - - - - - - - - - - - - - - - - - - - - - - - - - - - - - - -
\providecommand{\HeadFootShadeTriangleB}{}%
\renewcommand{\HeadFootShadeTriangleB}[4]{% 1: BL  2: BR  3: Top  4: Steps
    \tikzmath{%
        \opacstep=1/(#4+1);%
        \scalestep=1/#4;%
        \scalecur=1;%
    }%
    \foreach\i in {1,...,#4}{%
        \tikzmath{%
            \smoothen=1/(sqrt(\i));%
            \opacCur=\i*\opacstep*\smoothen;%
            \scalecur=(\i-1)*\scalestep;%
        }%
        \path[fill=FooterBG,line width=0pt,opacity=\opacCur]let\p1=($(#1)!\scalecur!(#2)$),\p2=($(#3)!\scalecur!(#2)$)in%
            (\p1)--(#2)--(\p2)--cycle;%
    }%
    \let\scalestep\undefined\let\opacstep\undefined\let\scalecur\undefined\let\smoothen\undefined\let\opacCur\undefined%
}%
\providecommand{\HeadFootShadeTriangleBR}{}%
\renewcommand{\HeadFootShadeTriangleBR}[3]{%
    \HeadFootShadeTriangleB{#1}{#2}{#3}{20}%
}%
\providecommand{\HeadFootShadeTriangleBL}{}%
\renewcommand{\HeadFootShadeTriangleBL}[3]{%
    \HeadFootShadeTriangleB{#2}{#1}{#3}{20}%
}%
%
\providecommand{\HeadFootShadeTriangleBInverse}{}%
\renewcommand{\HeadFootShadeTriangleBInverse}[4]{% 1: L  2: R  3: Top  4: Steps
    \tikzmath{%
        \opacstep=1/(#4+1);%
        \scalestep=1/#4;%
        \scalecur=1;%
    }%
    \foreach\i in {1,...,#4}{%
        \tikzmath{%
            \smoothen=1/(sqrt(\i));%
            \opacCur=\i*\opacstep*\smoothen;%
            \scalecur=(\i-1)*\scalestep;%
        }%
        \path[fill=FooterBG,line width=0pt,opacity=\opacCur]%
            let\p1=($(#3)!\scalecur!(#2)$),\p2=($(#3)!\scalecur!(#1)$)in%
            (\p1)--(#2)--(#1)--(\p2)--cycle;%
    }%
    \let\scalestep\undefined\let\opacstep\undefined\let\scalecur\undefined\let\smoothen\undefined\let\opacCur\undefined%
}%
\providecommand{\HeadFootShadeTriangleBInverseTopL}{}%
\renewcommand{\HeadFootShadeTriangleBInverseTopL}[3]{% 1: LB  2: RT  3: (L)Top
    \HeadFootShadeTriangleBInverse{#1}{#2}{#3}{20}%
}%
\providecommand{\HeadFootShadeTriangleBInverseTopR}{}%
\renewcommand{\HeadFootShadeTriangleBInverseTopR}[3]{% 1: LT  2: RB  3: (T)Top
    \HeadFootShadeTriangleBInverse{#2}{#1}{#3}{20}%
}%
%
%
%
% Not used currently, also not quite yet finished
\providecommand{\HeadFootShadeRectangleBR}{}%
\renewcommand{\HeadFootShadeRectangleBR}[3]{% 1: TL  2: BR  3: Steps
    \tikzmath{%
        \opacstep=1/(#3+1);%
        \scalestep=1/#3;%
        \scalecur=1;%
    }%
    \foreach\i in {1,...,#3}{%
        \tikzmath{%
            \smoothen=1/(sqrt(\i));%
            \opacCur=\i*\opacstep*\smoothen;%
            \scalecur=(\i-1)*\scalestep;%
        }%
        \path[fill=orange,line width=0pt,opacity=\opacCur]%
            let\p1=(#1-|#2),\p2=(#1|-#2),%
            \p3=($(\p1)!\scalecur!(#1)$),\p4=($(\p2)!\scalecur!(#1)$)in%
            (\p3)--(\p1)|-(\p2)--(\p4)--cycle;%
    }%
    \let\scalestep\undefined\let\opacstep\undefined\let\scalecur\undefined\let\smoothen\undefined\let\opacCur\undefined%
}%
%
%
%
%
%
% NONE of the above are actually used anymore. Exclusively this one below.
\DeclareDocumentCommand{\HeadFootShadeCorner}{%
    % Define the Coordinates to use before accordingly
    %  footShadeCorner_F[x]: A Fixed-Point
    %  footShadeCorner_M[x]: A Moving-Point
    O{20}% 1: Number of Shading Steps
}{%
    \tikzmath{%
        \opacstep=1/(#1+1);%
        \scalestep=1/#1;%
        \scalecur=1;%
    }%
    \foreach\i in {0,...,#1}{%
        \tikzmath{%
            \smoothen=1/(sqrt(\i+1));%
            \opacCur=(\i+1)*\opacstep*\smoothen;%
            \scalecur=\i*\scalestep;%
        }%
        \path[fill=FooterBG,line width=0pt,opacity=\opacCur]%
            let\p1=($(footShadeCorner_BottomMov)!\scalecur!(footShadeCorner_BottomFix)$),\p2=($(footShadeCorner_Mid1Mov)!\scalecur!(footShadeCorner_Mid1Fix)$),\p3=($(footShadeCorner_TopMov)!\scalecur!(footShadeCorner_TopFix)$)in%
            (footCornerAnch)--(\p1)--(\p2)to[out=\AngleAOffsetOut 10,in=\AngleAOffsetIn 15](\p3)--cycle;%
    }%
    \let\scalestep\undefined\let\opacstep\undefined\let\scalecur\undefined\let\smoothen\undefined\let\opacCur\undefined%
}%
\providecommand{\HeadFootShadeCornerTopRight}{}%
\renewcommand{\HeadFootShadeCornerTopRight}{%
\def\AngleAOffsetOut{}%
\def\AngleAOffsetIn{270-}%
\HeadFootShadeCorner%
}%
\providecommand{\HeadFootShadeCornerTopLeft}{}%
\renewcommand{\HeadFootShadeCornerTopLeft}{%
\def\AngleAOffsetOut{180-}%
\def\AngleAOffsetIn{270+}%
\HeadFootShadeCorner%
}%
%
%
%
%
%========================================================================
%       Layer Declaration & Assignment
% - - - - - - - - - - - - - - - - - - - - - - - - - - - - - - - - - - - -
\DeclareLayer[%
background,%
mode=picture,%
oddpage,%
foot,%
contents={%
  \put(\LenToUnit{\layerwidth},\LenToUnit{\layerheight}){\tikzmark{footRBAnchor}}%
  \begin{tikzpicture}[overlay,remember picture]%
  \coordinate(footBGAnchR)at($(pic cs:footRBAnchor)+(0,5pt)$);%
  \coordinate(footBgR_BR)at(current page.south east);%
  \coordinate(footBgR_BL)at($(footBGAnchR|-footBgR_BR)+(-0.4*\textwidth,0)$);%
  \coordinate(footBgR_RT)at($(footBGAnchR-|footBgR_BR)+(0,{0.4*(\paperwidth-\textwidth)})$);%
  \coordinate(footCornerR_TL)at(footBGAnchR);%
  \coordinate(footCornerR_TR)at(footBGAnchR-|footBgR_BR);%
  \coordinate(footCornerR_BL)at(footBGAnchR|-footBgR_BR);%
  %
  \coordinate(footCornerAnch)at(footBgR_BR);%
  \coordinate(footShadeCorner_BottomFix)at(footCornerR_BL);%
  \coordinate(footShadeCorner_BottomMov)at(footBgR_BL);%
  \coordinate(footShadeCorner_Mid1Fix)at(footShadeCorner_BottomFix);%
  \coordinate(footShadeCorner_Mid1Mov)at(footCornerR_TL);%
  \coordinate(footShadeCorner_TopFix)at(footCornerR_TR);%
  \coordinate(footShadeCorner_TopMov)at(footBgR_RT);%
  \HeadFootShadeCornerTopRight%
  %\HeadFootShadeTriangleBR{footBgR_BL}{footCornerR_BL}{footCornerR_TL}%
  %\HeadFootShadeTriangleBR{footCornerR_TL}{footCornerR_TR}{footBgR_RT}%
  %\HeadFootShadeTriangleBInverseTopL{footCornerR_BL}{footCornerR_TR}{footCornerR_TL}%
  %\path[draw=none,line width=0pt,fill=FooterBG](footCornerR_BL)--(footCornerR_TR)|-cycle;%
  \end{tikzpicture}%
}%
]{footBG.odd}%
\DeclareLayer[%
background,%
mode=picture,%
evenpage,%
foot,%
contents={%
  \put(0,\LenToUnit{\layerheight}){\tikzmark{footLBAnchor}}%
  \begin{tikzpicture}[overlay,remember picture]%
  \coordinate(footBGAnchL)at($(pic cs:footLBAnchor)+(0,5pt)$);%
  \coordinate(footBgL_BL)at(current page.south west);%
  \coordinate(footBgL_BR)at($(footBGAnchL|-footBgL_BL)+(0.4*\textwidth,0)$);%
  \coordinate(footBgL_LT)at($(footBGAnchL-|footBgL_BL)+(0,{0.4*(\paperwidth-\textwidth)})$);%
  \coordinate(footCornerL_TR)at(footBGAnchL);%
  \coordinate(footCornerL_TL)at(footBGAnchL-|footBgL_BL);%
  \coordinate(footCornerL_BR)at(footBGAnchL|-footBgL_BL);%
  %
  \coordinate(footCornerAnch)at(footBgL_BL);%
  \coordinate(footShadeCorner_BottomFix)at(footCornerL_BR);%
  \coordinate(footShadeCorner_BottomMov)at(footBgL_BR);%
  \coordinate(footShadeCorner_Mid1Fix)at(footShadeCorner_BottomFix);%
  \coordinate(footShadeCorner_Mid1Mov)at(footCornerL_TR);%
  \coordinate(footShadeCorner_TopFix)at(footCornerL_TL);%
  \coordinate(footShadeCorner_TopMov)at(footBgL_LT);%
  \HeadFootShadeCornerTopLeft%
  \end{tikzpicture}%
}%
]{footBG.even}%
\AddLayersToPageStyle{scrheadings}{footBG.odd,footBG.even}%
\AddLayersToPageStyle{plain.scrheadings}{footBG.odd,footBG.even}%
% % % % %
\DenKrKOMALayerTrackFootAdd{scrheadings}{footBG.odd,footBG.even}%
\DenKrKOMALayerTrackFootAdd{plain.scrheadings}{footBG.odd,footBG.even}%