% !TEX root = ../1main.tex
%

\begin{frame}
%This is the sample File for 'Slides / aka Presentation' Documents (i.e. a Slide-set, usually landscape format for giving a Presentation)
This is the sample File for \enquote*{Slides / aka Presentation} Documents (i.e. a Slide-set, usually landscape format for giving a Presentation)
\end{frame}


% \input{"\DenKrLayoutMainRootDir/8templates/example_chapter/3110_exampleSubject".tex}%


%______________________________________________________________
%\\\\\\\\\\\\\\\\\\\\\\\\\\\\\\\\\\\\\\\\\\\\\\\\\\\\\\\\\\\\\\\\\\\\......
%%%%%%%%%%%%%%%%%%%%%%%%%%%%%%%%%%%%%%%%%%%%%%%%%%%%%%%%%%%%%%%%%%%%%%%%%%%%%%%%%
%%#############################################################################%%
%    Preface       %%##########################################################%%
%%#############################################################################%%
%%%%%%%%%%%%%%%%%%%%%%%%%%%%%%%%%%%%%%%%%%%%%%%%%%%%%%%%%%%%%%%%%%%%%%%%%%%%%%%%%
%////////////////////////////////////////////////////////////////////´´´´´´
%''''''''''''''''''''''''''''''''''''''''''''''''''''''''''''''

\section{Introduction}
\label{sec:intro}


\begin{frame}
\frametitle{Introduction}
Text
\end{frame}







\part{Examples}

\section*{Table-Of-Contents}%
\begin{frame}[plain,allowframebreaks]%
\frametitle{Outline -- Examples}% ToC
\tableofcontents%
\end{frame}%

\section{Example Code}

\subsection{Arbitrary}

\begin{frame}
	\frametitle{Some Example Frame}
	Example-Citation:
	\cite{DenKr_denkrement1_indeco}
	% \nl%
	% A custom citation command that puts information about the reference to the footnote on first occurrence:
	% \citeff{DenKr_MigArb,DenKr_denkrement1_indeco}%

	\npi%
	Some special characters:
	»«.
	\nl%
	\gerguiquote{Done with Macro}\nl
	\enquote{Ordinary Quotation}\ \ \enquote*{Single Quotation}
	\nl%
	Example Acronym, Glossary-Entry \& Symbol:\nl
	\gls{fsa}, \glsunset{api}\gls{api}\nl
	\gls{molmass}, \gls{sigma}\nl
	\gls{eventfd}
\end{frame}



\begin{frame}
	\frametitle{Lists}
	\begin{minipage}[t]{0.48\linewidth}
		\begin{itemize}
		\item Example Itemization
			\begin{itemize}
			\item%
				Lvl-2
				\begin{itemize}
				\item%
					Lvl-3
				\end{itemize}
			\end{itemize}
		\end{itemize}
	\end{minipage}
	%
	\hfill
	%
	\begin{minipage}[t]{0.48\linewidth}
		\begin{enumerate}
		\item Example Enumeration
			\begin{enumerate}
			\item%
				Lvl-2
				\begin{enumerate}
				\item%
					Lvl-3
				\end{enumerate}
			\end{enumerate}
		\end{enumerate}
	\end{minipage}
	
	\begin{description}
	\item[Example Description List]%
		Just some random Text without any sense, but with sufficient length to cause a line-break, so that the indentation is properly showing influence.
	\item[2nd Item] Some Text, just barely long enough, to cause a linebreak, to see in the indentation.
	\end{description}
\end{frame}




\subsection{Math}

\providecommand{\showcaseMathString}{}%
\begin{frame}
	\frametitle{Math Cmds \& Font}
	\only<1>{
	\newcommand{\I}{\mathrm{i}}
	$y = \int_0^x\cos(x)\,\mathrm{d}{x} = \frac{e^{\I x} - e^{-\I x}}{2\I} | 0123456789 \geq i \leqslant l$
	\nl
	\begin{equation}
	\oint \iint \sum_{k=0}^{m} \alpha \beta \gamma \delta \epsilon + \max \left\{ \frac{\| v\oplus w\bigoplus z \Vert}{x^{\mathbf{N} \varepsilon \symbb{C}^{N\times 10}}} \right\}
	\end{equation}\nl
	QED Variants: $\Box \square \blacksquare ∎ \qedsym \char"220E$
	\npi
	\renewcommand{\showcaseMathString}{eix \Rightarrow 0123 \geq}%
	\begin{tabular}{lll}%
		plain&\texttt{\lstTEXargumentBO{$ $}}&$\showcaseMathString$\\
		Upright&\texttt{\lstTEXmacro{mathrm}\{\}}&$\mathrm{\showcaseMathString}$\\ % Or \mathup{}
		Italic&\texttt{\lstTEXmacro{mathit}\{\}}&$\mathit{\showcaseMathString}$\\
		Bold&\texttt{\lstTEXmacro{mathbf}\{\}}&$\mathbf{\showcaseMathString}$\\
		Sans-Serif&\texttt{\lstTEXmacro{mathsf}\{\}}&$\mathsf{\showcaseMathString}$\\
		Monospaced-Teletype&\texttt{\lstTEXmacro{mathtt}\{\}}&$\mathtt{\showcaseMathString}$\\
	\end{tabular}
	}

	\only<2>{
	\renewcommand{\showcaseMathString}{CDEIJUV 012}%
	\begin{tabular}{lll}%
		\multicolumn{3}{l}{$\symbb{ABCDEFGHIJKLMNOPQRSTUVWXYZ 0123456789}$}\\
		\multicolumn{3}{l}{$\symbbit{ABCDEFGHIJKLMNOPQRSTUVWXYZ 0123456789}$}\\
		&&\\
		% Caligrahpic, Script, Fraktal
		$\symcal{\showcaseMathString}$&$\symscr{\showcaseMathString}$&$\symfrak{\showcaseMathString}$\\
		$\symbfcal{\showcaseMathString}$&$\symbfscr{\showcaseMathString}$&$\symbffrak{\showcaseMathString}$\\
		&&\\
		$\symsfup{\showcaseMathString}$&$\symsfit{\showcaseMathString}$&\\
		$\symbfup{\showcaseMathString}$&$\symbfit{\showcaseMathString}$&\\
		$\symbfsfup{\showcaseMathString}$&$\symbfsfit{\showcaseMathString}$&$\symbfsf{\showcaseMathString}$\\
	\end{tabular}
	}
\end{frame}


% \begin{frame}
% 	\frametitle{This is the second slide}
% 	\framesubtitle{A bit more information about this}
% 	More content goes here
% \end{frame}


% \begin{frame}{}
% 	\frametitle{Einführung}
% 	Entwickelt wurde das sogenannte >>Abstract Interface (AI)<<
% 	\begin{itemize}%
% 	  \item Aufgaben
% 		\begin{itemize}%
% 		  \item Verwaltung der physikalischen Netzwerkschnittstellen
% 		  \item >>Seamless Handover<< (dt: Bruchloser/Unterbrechungsfreier
% 		  Funkkanal-Wechsel)
% 		  \item Verbindungs-Optimierung
% 		\end{itemize}
% 	  \item Eigenschaften
% 		\begin{itemize}%
% 		  \item C-Programm
% 			\begin{itemize}%
% 			  \item Name: AbsInt
% 			\end{itemize}
% 		  \item Linux
% 		\end{itemize}
% 	\end{itemize}
% \end{frame}



% \begin{frame}{}
% 	\frametitle{Aufbau des Vortrags}
% 	\begin{itemize}%
% 	  \item Einleitung
% 		\begin{itemize}%
% 		  \item Eingrenzung von Thema und Ziel
% 		  \item Schwerpunkt der Arbeit
% 		  \item Stand der Technik
% 		\end{itemize}
% 	  \item Theorie
% 		\begin{itemize}%
% 		  \item Software-Defined Switching
% 		\end{itemize}
% 	  \item Abstract Interface
% 		\begin{itemize}%
% 		  \item Unterbrechungsfreier WLAN-Kanal-Wechsel
% 		  \item WLAN-Monitor
% 		\end{itemize}
% 	  \item Evaluierung
% 		\begin{itemize}%
% 		  \item Testreihen
% 		  \item Fazit
% 		\end{itemize}
% 	\end{itemize}
% \end{frame}



\begin{frame}
\frametitle{Problem Statement \& Goal}
	Distributed System
	Communicating Applications
	-> Deployment and Management of Apps
	-> Efficient Communication

	So I introduced a couple new features
	-> Also kept the entire system working as efficient as possible
\end{frame}










%______________________________________________________________
%\\\\\\\\\\\\\\\\\\\\\\\\\\\\\\\\\\\\\\\\\\\\\\\\\\\\\\\\\\\\\\\\\\\\......
%%%%%%%%%%%%%%%%%%%%%%%%%%%%%%%%%%%%%%%%%%%%%%%%%%%%%%%%%%%%%%%%%%%%%%%%%%%%%%%%%
%%#############################################################################%%
%    Actual Content       %%###################################################%%
%%#############################################################################%%
%%%%%%%%%%%%%%%%%%%%%%%%%%%%%%%%%%%%%%%%%%%%%%%%%%%%%%%%%%%%%%%%%%%%%%%%%%%%%%%%%
%////////////////////////////////////////////////////////////////////´´´´´´
%''''''''''''''''''''''''''''''''''''''''''''''''''''''''''''''
















%______________________________________________________________
%\\\\\\\\\\\\\\\\\\\\\\\\\\\\\\\\\\\\\\\\\\\\\\\\\\\\\\\\\\\\\\\\\\\\......
%%%%%%%%%%%%%%%%%%%%%%%%%%%%%%%%%%%%%%%%%%%%%%%%%%%%%%%%%%%%%%%%%%%%%%%%%%%%%%%%%
%%#############################################################################%%
%    Postface       %%#########################################################%%
%%#############################################################################%%
%%%%%%%%%%%%%%%%%%%%%%%%%%%%%%%%%%%%%%%%%%%%%%%%%%%%%%%%%%%%%%%%%%%%%%%%%%%%%%%%%
%////////////////////////////////////////////////////////////////////´´´´´´
%''''''''''''''''''''''''''''''''''''''''''''''''''''''''''''''

\section{Summary, Conclusion, Future Work}
\label{sec:Concl}




% This command serves to balance the column lengths on the last page of the document manually.
% It shortens the textheight of the last page by a suitable amount.
% This command does not take effect until the next page so it should come on the page before the last.
% Make sure that you do not shorten the textheight too much.
%\DenKrLastPageColumnBalancing{1cm}%
%
%
% use section* for acknowledgment
%#################################################################
%##=========================================================######
%##---------------------------------------------------------######
% \section*{Acknowledgment}%
%##=========================================================######
%#################################################################

% \begin{frame}[plain]
% \frametitle{Acknowledgment}
% 	The authors acknowledge the financial support by the German \textit{Federal Ministry of Education and Research (BMBF)} within the project »Open6GHub« \{16KISK003K\}.
% 	% \par\noindent The list of authors is arranged in alphabetical order.
% 	%
% 	% This work is supported by the German \textit{Federal Ministry of Education and Research (BMBF)} within the project »Open 6G Hub« \{16KIS1158K\}.
% \end{frame}
