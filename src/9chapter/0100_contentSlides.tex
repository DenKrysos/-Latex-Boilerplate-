% !TEX root = ../1main.tex
%

\vspace{\baselineskip}
%This is the sample File for 'Slides / aka Presentation' Documents (i.e. a Slide-set, usually landscape format for giving a Presentation)
This is the sample File for \enquote*{Slides / aka Presentation} Documents (i.e. a Slide-set, usually landscape format for giving a Presentation)
\vspace{\baselineskip}


% \input{"\DenKrLayoutMainRootDir/8templates/example_chapter/3110_exampleSubject".tex}%


%______________________________________________________________
%\\\\\\\\\\\\\\\\\\\\\\\\\\\\\\\\\\\\\\\\\\\\\\\\\\\\\\\\\\\\\\\\\\\\......
%%%%%%%%%%%%%%%%%%%%%%%%%%%%%%%%%%%%%%%%%%%%%%%%%%%%%%%%%%%%%%%%%%%%%%%%%%%%%%%%%
%%#############################################################################%%
%    Preface       %%##########################################################%%
%%#############################################################################%%
%%%%%%%%%%%%%%%%%%%%%%%%%%%%%%%%%%%%%%%%%%%%%%%%%%%%%%%%%%%%%%%%%%%%%%%%%%%%%%%%%
%////////////////////////////////////////////////////////////////////´´´´´´
%''''''''''''''''''''''''''''''''''''''''''''''''''''''''''''''

\section{Introduction}
\label{sec:intro}







\section{Related Work}
\label{sec:RelWork}


%
% Title Page
\begin{frame}[plain]%
\titlepage%
\end{frame}%
% \pagestyle{plain}
%
% Präsentation
%
\part{Hauptteil}
\begingroup\AtBeginSection[]{}
\section{Einleitung}%
\endgroup
\subsection{Einführung}%


\begin{frame}{Test 1}
	\frametitle{This is the first slide}
	Content goes here
\end{frame}


\providecommand{\showcaseMathString}{}%
\begin{frame}{Math}
	\frametitle{Math Cmds \& Font}
	\only<1>{
	\newcommand{\I}{\mathrm{i}}
	$y = \int_0^x\cos(x)\,\mathrm{d}{x} = \frac{e^{\I x} - e^{-\I x}}{2\I} | 0123456789 \geq i \leqslant l$
	\nl
	\begin{equation}
	\oint \iint \sum_{k=0}^{m} \alpha \beta \gamma \delta \epsilon + \max \left\{ \frac{\| v\oplus w\bigoplus z \Vert}{x^{\mathbf{N} \varepsilon \symbb{C}^{N\times 10}}} \right\}
	\end{equation}\nl
	QED Variants: $\Box \square \blacksquare ∎ \qedsym \char"220E$
	\npi
	\renewcommand{\showcaseMathString}{eix \Rightarrow 0123 \geq}%
	\begin{tabular}{lll}%
		plain&\texttt{\lstTEXargumentBO{$ $}}&$\showcaseMathString$\\
		Upright&\texttt{\lstTEXmacro{mathrm}\{\}}&$\mathrm{\showcaseMathString}$\\ % Or \mathup{}
		Italic&\texttt{\lstTEXmacro{mathit}\{\}}&$\mathit{\showcaseMathString}$\\
		Bold&\texttt{\lstTEXmacro{mathbf}\{\}}&$\mathbf{\showcaseMathString}$\\
		Sans-Serif&\texttt{\lstTEXmacro{mathsf}\{\}}&$\mathsf{\showcaseMathString}$\\
		Monospaced-Teletype&\texttt{\lstTEXmacro{mathtt}\{\}}&$\mathtt{\showcaseMathString}$\\
	\end{tabular}
	}

	\only<2>{
	\renewcommand{\showcaseMathString}{CDEIJUV 012}%
	\begin{tabular}{lll}%
		\multicolumn{3}{l}{$\symbb{ABCDEFGHIJKLMNOPQRSTUVWXYZ 0123456789}$}\\
		\multicolumn{3}{l}{$\symbbit{ABCDEFGHIJKLMNOPQRSTUVWXYZ 0123456789}$}\\
		&&\\
		% Caligrahpic, Script, Fraktal
		$\symcal{\showcaseMathString}$&$\symscr{\showcaseMathString}$&$\symfrak{\showcaseMathString}$\\
		$\symbfcal{\showcaseMathString}$&$\symbfscr{\showcaseMathString}$&$\symbffrak{\showcaseMathString}$\\
		&&\\
		$\symsfup{\showcaseMathString}$&$\symsfit{\showcaseMathString}$&\\
		$\symbfup{\showcaseMathString}$&$\symbfit{\showcaseMathString}$&\\
		$\symbfsfup{\showcaseMathString}$&$\symbfsfit{\showcaseMathString}$&$\symbfsf{\showcaseMathString}$\\
	\end{tabular}
	}

	Quality, 0123 fissure
\end{frame}


% \begin{frame}
% 	\frametitle{This is the second slide}
% 	\framesubtitle{A bit more information about this}
% 	More content goes here
% \end{frame}


% \begin{frame}{}
% 	\frametitle{Einführung}
% 	Entwickelt wurde das sogenannte >>Abstract Interface (AI)<<
% 	\begin{itemize}%
% 	  \item Aufgaben
% 		\begin{itemize}%
% 		  \item Verwaltung der physikalischen Netzwerkschnittstellen
% 		  \item >>Seamless Handover<< (dt: Bruchloser/Unterbrechungsfreier
% 		  Funkkanal-Wechsel)
% 		  \item Verbindungs-Optimierung
% 		\end{itemize}
% 	  \item Eigenschaften
% 		\begin{itemize}%
% 		  \item C-Programm
% 			\begin{itemize}%
% 			  \item Name: AbsInt
% 			\end{itemize}
% 		  \item Linux
% 		\end{itemize}
% 	\end{itemize}
% \end{frame}



% \begin{frame}{}
% 	\frametitle{Aufbau des Vortrags}
% 	\begin{itemize}%
% 	  \item Einleitung
% 		\begin{itemize}%
% 		  \item Eingrenzung von Thema und Ziel
% 		  \item Schwerpunkt der Arbeit
% 		  \item Stand der Technik
% 		\end{itemize}
% 	  \item Theorie
% 		\begin{itemize}%
% 		  \item Software-Defined Switching
% 		\end{itemize}
% 	  \item Abstract Interface
% 		\begin{itemize}%
% 		  \item Unterbrechungsfreier WLAN-Kanal-Wechsel
% 		  \item WLAN-Monitor
% 		\end{itemize}
% 	  \item Evaluierung
% 		\begin{itemize}%
% 		  \item Testreihen
% 		  \item Fazit
% 		\end{itemize}
% 	\end{itemize}
% \end{frame}










%______________________________________________________________
%\\\\\\\\\\\\\\\\\\\\\\\\\\\\\\\\\\\\\\\\\\\\\\\\\\\\\\\\\\\\\\\\\\\\......
%%%%%%%%%%%%%%%%%%%%%%%%%%%%%%%%%%%%%%%%%%%%%%%%%%%%%%%%%%%%%%%%%%%%%%%%%%%%%%%%%
%%#############################################################################%%
%    Actual Content       %%###################################################%%
%%#############################################################################%%
%%%%%%%%%%%%%%%%%%%%%%%%%%%%%%%%%%%%%%%%%%%%%%%%%%%%%%%%%%%%%%%%%%%%%%%%%%%%%%%%%
%////////////////////////////////////////////////////////////////////´´´´´´
%''''''''''''''''''''''''''''''''''''''''''''''''''''''''''''''
















%______________________________________________________________
%\\\\\\\\\\\\\\\\\\\\\\\\\\\\\\\\\\\\\\\\\\\\\\\\\\\\\\\\\\\\\\\\\\\\......
%%%%%%%%%%%%%%%%%%%%%%%%%%%%%%%%%%%%%%%%%%%%%%%%%%%%%%%%%%%%%%%%%%%%%%%%%%%%%%%%%
%%#############################################################################%%
%    Postface       %%#########################################################%%
%%#############################################################################%%
%%%%%%%%%%%%%%%%%%%%%%%%%%%%%%%%%%%%%%%%%%%%%%%%%%%%%%%%%%%%%%%%%%%%%%%%%%%%%%%%%
%////////////////////////////////////////////////////////////////////´´´´´´
%''''''''''''''''''''''''''''''''''''''''''''''''''''''''''''''

\section{Conclusion and future work}
\label{sec:Concl}




% This command serves to balance the column lengths on the last page of the document manually.
% It shortens the textheight of the last page by a suitable amount.
% This command does not take effect until the next page so it should come on the page before the last.
% Make sure that you do not shorten the textheight too much.
%\DenKrLastPageColumnBalancing{1cm}%
%
%
% use section* for acknowledgment
%#################################################################
%##=========================================================######
%##---------------------------------------------------------######
\section*{Acknowledgment}%
%##=========================================================######
%#################################################################

% The authors acknowledge the financial support by the German \textit{Federal Ministry of Education and Research (BMBF)} within the project »Open6GHub« \{16KISK003K\}.
%\par\noindent The list of authors is arranged in alphabetical order.
%
%This work is supported by the German \textit{Federal Ministry of Education and Research (BMBF)} within the project »Open 6G Hub« \{16KIS1158K\}.
