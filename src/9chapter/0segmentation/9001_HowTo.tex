% !TEX root = ../1main.tex
%
\chapter*{HowTo}%
\addcontentsline{toc}{chapter}{HowTo}% Add to table of contents without numbering

Die wichtigsten Ordner \& Dateien für die Konsorten:
\begin{itemize}
	\item \texttt{./\textcolor{RedOrange}{9chapter}}
	\begin{itemize}
		\item Eine \enquote{\_.tex} Datei pro WP $\rightarrow$ Dort bitte Inhalte rein.
	\end{itemize}
	\item \texttt{./\textcolor{RedOrange}{8graphics}}
	\begin{itemize}
		\item Ein \textit{Directory} pro WP $\rightarrow$ Dort bitte einzubindende Bilder rein.
		\item \textit{Vorlage} zum Einbinden von Bildern (Root-Dir via Macro):\par
\begin{DenKrLst}[language=DenKr-C,morekeywords={[3]{begin,end,caption,centering,includegraphics,linewidth,DenKrGraphicsRootDir}},morekeywords={[6]{figure}},morekeywords={[7]{label}},]{}{}
\begin{figure}[!ht]
	\centering
	\includegraphics[width=0.7\linewidth]{\DenKrGraphicsRootDir/WP/ImgName.png}
	\caption{Caption}
	\label{fig:ImgLabel}
\end{figure}
\end{DenKrLst}
	\end{itemize}
	\item \texttt{./\textcolor{RedOrange}{1supply}}
	\begin{itemize}
		\item \textcolor{orange}{\textit{Akronyme}}. Wer mag kann Akronyme in die Datei \enquote{\textcolor{PineGreen}{\texttt{glossaries.tex}}} einfügen und anschließend benutzen.
			\begin{itemize}
				\item \textbackslash gls\{\ \}, \textbackslash glspl\{\ \}, \textbackslash glstext\{\ \}, \textbackslash glsdesc\{\ \}, etc.
			\end{itemize}
		\item \textcolor{orange}{\textit{Literatur}}: Eine Datei \enquote{\textcolor{PineGreen}{\texttt{literatur\_WPx.bib}}} pro WP. $\rightarrow$ Hineinkopieren, im Text mit Key referenzieren (just as you know the drill).
	\end{itemize}
\end{itemize}

\npi
\begin{itemize}
	\item Die Datei \enquote{\textcolor{PineGreen}{\texttt{./3Disposition.tex}}} enthält das grundlegende Datei-Inklusions-Skelett.
\end{itemize}




\section*{Compilation / Compiler (Local Compilation)}

Wer lokal kompilieren möchte:
\begin{itemize}
    \item Compiler: \texttt{\textcolor{DenKrColor_Violet_3}{LuaLaTeX}}
    \item (Empfohlene) Toolchain: \texttt{\textcolor{DenKrColor_Violet_3}{TeXLive-\textbf{2025}}}
    \item[-] - - - - - - - - - - -
    \item Git Funktion von Overleaf: Links oben \enquote*{\textit{Menu}} $\rightarrow$ \enquote*{\textit{Git}} $\rightarrow$ Git-Clone
    \item Das Projekt ist vorbereitet, einfach mit \texttt{\textcolor{DenKrColor_Violet_3}{VSCode}} kompiliert zu werden:
        \begin{itemize}
            \item Install Extension \enquote{\textit{\textcolor{DenKrColor_Violet_1}{LaTeX Workshop}}}
                \begin{itemize}
                    \item \textit{(Optional)} eventuell ebenso: \enquote{\textit{\textcolor{DenKrColor_Violet_1}{LaTeX Utilities}}}, \enquote{\textit{\textcolor{DenKrColor_Violet_1}{LTeX+ -- grammar/spell checking using LanguageTool}}}
                \end{itemize}
            \item In VSCode, links oben: \enquote*{File} $\rightarrow$ \enquote*{\textit{Open Workspace from File\ldots}}
                \begin{itemize}[labpragA]
                    \item Die Datei \enquote{\textcolor{PineGreen}{\texttt{$\ast$.code-workspace}}} im Root-Dir öffnen.
                    \item Compile mit \texttt{Ctrl + Enter}
                \end{itemize}
            \item Erfolgreiche Kompilierung erzeugt im Root-Dir die Directories \texttt{\textcolor{RedOrange}{/.1out\_pre/}}, \texttt{\textcolor{RedOrange}{/.2out\_final/}}, worin die erzeuge \texttt{.pdf} zu finden ist.
                \begin{itemize}
                    \item \textit{(Optional)}: Kleine, schnelle Anpassungen in den Dateien \enquote{\textcolor{PineGreen}{\texttt{.latexmkrc}}} \& \enquote{\textcolor{PineGreen}{\texttt{.latexmk\_post}}} erlauben, die \textit{Funktion} zu \textit{aktivieren}, dass das kompilierte \textit{Resultat} in das Root-Dir \textit{kopiert} und \textit{umbenannt} wird.
                    \item Ist in den Dateien mittels Kommentaren dokumentiert. \textit{tl;dr}:
                        \begin{itemize}
                            \item \enquote{\textcolor{PineGreen}{\texttt{.latexmkrc}}} definiert finalen \textit{Dateinamen} \& \textit{Speicherort}:\nl
                                \begin{tabbing}
                                    \labpragCsym\ \texttt{\lstTEXFontLibrary{\$resultName}=\lstTEXFontMacro{'1compiled'};}\hspace{1.5em}\=$\rightarrow$\hspace{1em}\=\textit{<\textbf{FileName}>}.pdf\\
                                    \labpragCsym\ \texttt{\lstTEXFontLibrary{\$resultSubDir}=\lstTEXFontMacro{''};}\hspace{1.5em}\>$\rightarrow$\hspace{1em}\>\textit{<\textbf{Directory}>} to \textit{copy to}, \textit{relative} to Project-Root-Dir.
                                \end{tabbing}
                            \item \enquote{\textcolor{PineGreen}{\texttt{.latexmk\_post}}} aktiviert die Funktion:
                                \begin{itemize}
                                    \item[\labpragCsym] Remove the Comment-Sign (\texttt{\textcolor{eclipse_1_comment}{\#}}) in front of the Line \texttt{\textcolor{eclipse_1_comment}{\#cpy\_final(\$thisAbsPath,[\ldots]}}
                                \end{itemize}
                        \end{itemize}
                \end{itemize}
        \end{itemize}
    \item[-] - - - - - - - - - - -
    \item (Falls Kompilierung lokal nicht funktioniert, der General-Tipp: \textit{Update} Toolchain)
\end{itemize}



\section*{Working-State -- temporary Elements}

\begin{itemize}
    \item Die Datei \enquote{\textcolor{PineGreen}{\texttt{./2ProjektSetup.tex}}} enthält das Macro \enquote{\textcolor{RoyalBlue}{\texttt{\textbackslash DenKrCommentsUsage}}}. Dies bietet zwei Hauptfunktionen, welche on/off-toggled werden können:
    \begin{enumerate}
        \item Eine Inline-Kommentar-Funktion. (Siehe Letzte Seite des Dokuments.) Makro auf \texttt{'1'} gesetzt aktiviert diese. Makro auf \texttt{'0'} deaktiviert (und entfernt alle Kommentare aus kompilierter \textit{.pdf}).
        \item Erlaubt es auch, Elemente zu definieren, die temporär im kompilierten Resultat erscheinen sollen und einfach deaktiviert werden können.
        \begin{itemize}
            \item Man umschließe etwas mit \texttt{\textcolor{Bittersweet}{\textbackslash disablewr}\{\}}
            \item Ist besagtes Setting-Macro auf \texttt{'1'} gesetzt, erscheinen die Inhalte. Ist es auf \texttt{'0'} gesetzt, erscheinen sie nicht.
        \end{itemize}
    \end{enumerate}
    \item[\labpragAsym] Ihr könnt euch also während der Bearbeitung Kommentare, Notizen oder vorübergehend einzubauende Passagen hinterlassen, die während Bearbeitung zu sehen sind.
    \begin{itemize}
        \item Kompiliert Marc abschließend das finale Dokument, muss er nur \enquote{\textcolor{RoyalBlue}{\texttt{\textbackslash DenKrCommentsUsage}}} auf \texttt{'0'} setzen und all dies erscheint nicht mehr in der kompilierten .pdf.
    \end{itemize}
    \item (Dieses HowTo hier ist übrigens genau so eingebaut und verschwindet, setzt man besagten Wert auf \texttt{'0'}.)
\end{itemize}




\section*{Typografie, Typesetting, Optik}

Wer es richtig fancy treiben möchte:
\begin{itemize}[labpragA]
    \item Im Ordner \texttt{./\textcolor{RedOrange}{2Info}} liegen Datein zur Präsentation bereit, welche Werkzeuge zur hübschen Formatierung bereitgestellt sind.
        \begin{itemize}
            \item \enquote{\textcolor{PineGreen}{\texttt{./1\_main\_3\_SHOWCASE\_components.pdf}}}: Einzelne Typesetting Komponenten
            \item \textit{Andere}: Grundlegende, Dokument-weite Layout/Typografie.
        \end{itemize}
\end{itemize}




% \vspace*{1\baselineskip}
\section*{(Optional) Customization}
\noindent
Nebensächliche Option. Visuelle Erscheinung des Dokuments.
\begin{itemize}
	\item In der Datei \enquote{\textcolor{PineGreen}{./1supply/OptionalDocConfig/OverwriteProperties/typografie\_colors.tex}} kann das Farbschema verändert werden.
	\item In der Datei \enquote{\textcolor{PineGreen}{./1supply/OptionalDocConfig/DISABLED\_typografie\_pageLayout.tex}} kann das Aussehen von Page-Layout Komponenten angepasst werden (Pager-Header/Footer, Chapter/Section/\ldots Heading).
	\begin{itemize}
		\item (Rename to remove the prefixed \enquote{\texttt{DISABLED\_}} if used.)
	\end{itemize}
\end{itemize}



