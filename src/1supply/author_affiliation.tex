%________________________________
%<><><><><><><><><><><><><><><><>
%		Authors
%vvvvvvvvvvvvvvvvvvvvvvvvvvvvvvvv
\DenKrAuthor{1}{Dennis Krummacker}[dennis.krummacker@dfki.de][0000-0001-9799-4870]% ORCID as second optional Argument. See below at USAGE Notes
\DenKrAuthor{1}{FirstA LastA}[firstA.lastA@company.de]%
\DenKrAuthor{2}{FirstB LastB}[lastB@corporate.org ° !]% See below at USAGE Notes
\DenKrAuthorNewline%
\DenKrAuthor{1°3}{FirstC MidC. LastC}[firstC_midC.lastC@domain.com]%
\DenKrAuthor{2°3}{Prof. Smarty}[smarty.pants@institute1.com@2°smarty@uni.de@3]%
%\DenKrAuthor{1°2}{Hans D. Schotten}[hans_dieter.schotten@dfki.de@1 ° schotten@rptu.de@2]%
%^^^^^^^^^^^^^^^^^^^^^^^^^^^^^^^^
%
%
%
%________________________________
%<><><><><><><><><><><><><><><><>
%		Affiliations
%vvvvvvvvvvvvvvvvvvvvvvvvvvvvvvvv
\DenKrAffiliation{1}{
	% Intelligent Networks Research Group,
	German Research Center for Artificial Intelligence (DFKI),
	% D-67663 Kaiserslautern,
	% D-Kaiserslautern.%
	Germany.%
}{\{firstname\}.\{lastname\}@dfki.de}%
%
\DenKrAffiliation{2}{
	%WICON,
	% Institute for Wireless Communication and Navigation,
	University of Kaiserslautern (RPTU),
	% D-67663 Kaiserslautern,
	Germany.%
}{\{lastname\}@rptu.de}%
%
\DenKrAffiliation{3}{
	Company Co., Ltd, D-10115 Berlin.
}{
	some.person@company.com
}%
\DenKrAffiliation{4}{aaaaAffil}{mail@mail.org}
\DenKrAffiliation{5}{bbbbAffil}{}
\DenKrAffiliation{}{ccccAffil}{a_a@a.org°b@b.org}
\DenKrAffiliation{5°4}{Institute-Address}{person@company.org ° boss@company.org}% See below at USAGE Notes
%^^^^^^^^^^^^^^^^^^^^^^^^^^^^^^^^
%
%
%
%
%
%________________________________________________________________________
%------------------------------------------------------------------------
%				(Optional) Configuration
%/\/\/\/\/\/\/\/\/\/\/\/\/\/\/\/\/\/\/\/\/\/\/\/\/\/\/\/\/\/\/\/\/\/\/\/\
% In which convention Mail-addresses are printed.
%   Sometimes desired is to specify a uniform format per company, like {firstname.lastname}@company.com
%   Sometimes you want to explicitly each single mail-address per author
% - collective
%     Uses the Mail specification from \DenKrAffiliation
% - individual
%     Uses the Mail specifications from \DenKrAuthor *partly*.
%     - This only takes the text before the @ and aggregates them with the domain from the affiliation.
% - single   [TODO]
%     Uses the Mail specifications from \DenKrAuthor *entirely*.
%     This just directly takes the Email-Addresses from the Author.
% - auto
%     Makes it depending on the selected document format
\renewcommand{\DenKrAuthorMailSelect}{auto}%
%
% IEEE Form
%  In case, the IEEEtran documentclass is used: In which way the authors section shall be formatted
% - vertical
%     Everything put together. First a list of the Author's names, then vertically below, the affiliations.
% - compact
%     Same as 'vertical', but without a separating blank line before the affiliations.
% - horizontal
%     Several distinct blocks, which are horizontally sequentially aligned
%     May work well for up to ~3 different affiliations
\renewcommand{\DenKrAuthorIEEEForm}{vertical}%
%
% Linebreaks to comply to Textwidth
% - Set the following to '1' and you do not necessarily need to specify Linebreaks above (but you lose a certain control).
\renewcommand{\DenKrAuthorParboxed}{0}%
%
% ORCID: Whether or not it shall be printed (In case specified)
% - '0' NOT   |   '1' Added
% - (Of course, simply not specifying any above - as they are optional arguments - would also just not do anything in that regard)
\renewcommand{\DenKrAuthorORCIDSwitch}{1}%
%/\/\/\/\/\/\/\/\/\/\/\/\/\/\/\/\/\/\/\/\/\/\/\/\/\/\/\/\/\/\/\/\/\/\/\/\
%				END Config
%------------------------------------------------------------------------
%========================================================================
%
%
%
%
%
%________________________________________________________________________
%------------------------------------------------------------------------
%				USAGE Arguments
%/\/\/\/\/\/\/\/\/\/\/\/\/\/\/\/\/\/\/\/\/\/\/\/\/\/\/\/\/\/\/\/\/\/\/\/\
% - \DenKrAuthor{Idx}{Name}[Mail-Address]
%    - Idx / Used as Ref-Mark & Association between Author and Affiliation:
%       - Ref-Mark: A Symbol for associating the Author with an Affiliation
%       - Association: Depending on the Layout, Author and Affil are grouped according to this.
%          - Likewise, when configured to take Mail-Addr indidivual per Author, but put to Affiliation
%    - Name: Name of the Author
%    - Mail-Addr: An Author's individual Mail-Address. Depending on set Layout and Configuration.
%       - Multiple Addresses can be specified per Author, via separating them with '°'
%       - When 'individual' Addrs are print, all Mail-Addrs of this list are put by default.
%       - This can be filtered via a second '@' after one Addr. With such, a single Addr is only associated with the Idx after the 'second @'.  [TODO]
% - \DenKrAffiliation{Idx}{Address}{Mail-Address}
%    - Idx / Used as Ref-Mark & Association between Author and Affiliation:
%       - [Like above]
%    - Address: Address of a Company
%    - Mail-Address: Mail Address specified per company. Probably specify a uniform convention for all employees (e.g. \{firstname\}.\{lastname\}@dfki.de) if applicable.
%________________________________________________________________________
%------------------------------------------------------------------------
%				More USAGE Notes
%/\/\/\/\/\/\/\/\/\/\/\/\/\/\/\/\/\/\/\/\/\/\/\/\/\/\/\/\/\/\/\/\/\/\/\/\
% - Don't be confused by the Degree-Sign (°). It is just used as Separator in Lists.
% - Append a "°!" to an E-Mail-Address in the 3rd (optional) argument of the \DenKrAuthor{}{}[] Cmd to force a linebreak after this Mail-Addr
%     -> taking effect in the case '\DenKrAuthorMailSelect' is set to 'individual'
% - ORCID: Can be given as 4th argument, which is a 2nd 'optional' argument in sequence: \DenKrAuthor{}{}[][]
%    - As such, it can only be specified, when also the (optional) E-Mail-Address argument is present.
%    - If you need to specify an ORCID, but NO E-Mail-Address: Give as E-Mail-Address [!], to specify it as 'No-Value / invalid / not-applicable / NA / empty'
%       -> \DenKrAuthor{1}{Name}[!][<ORCID>]
%/\/\/\/\/\/\/\/\/\/\/\/\/\/\/\/\/\/\/\/\/\/\/\/\/\/\/\/\/\/\/\/\/\/\/\/\
%				END Config
%------------------------------------------------------------------------
%========================================================================